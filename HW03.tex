\documentclass[12pt]{article}
\usepackage[top=1.0in, bottom=0.75in, left=0.75in, right=2.5in]{geometry}
\usepackage{fancyhdr, textpos, enumerate}
\usepackage{amsmath, amssymb, amsthm}
\usepackage{ebgaramond-maths, plex-mono}
\usepackage[makeroom]{cancel}

\pagestyle{fancy}
\fancyhf{}
\setlength\headheight{28pt}
\setlength\headwidth{7in}
\setlength{\TPHorizModule}{1pt}
\setlength{\TPVertModule}{1pt}
\lhead{\textsc{group theory\\spring 2018}}
\rhead{\textsc{homework bundle \#3\\page \thepage}}
\chead{\textsc{rob\\ireton}}

\newcommand{\zee}{\mathbb{Z}}

\newenvironment{exercise}[2]{\begin{textblock}{32}[1,0](0,#2)\noindent#1\end{textblock}}{\vspace{1in}}

\begin{document}
\begin{exercise}{1.47}{8}
	{\noindent}Let $G$ be a group and fix $a\in G$. Consider the function $\gamma_a:G\to G$ given by $\gamma_a(x)=axa^{-1}$. (This is called conjugation by $a$.)
	\begin{enumerate}[(i.)]
		\item Prove that $\gamma_a$ is an isomorphism.
		\bigskip
		\begin{proof}
			Let $x,y\in G$.
			\par
			$\gamma_a(xy)=axya^{-1}=ax1ya^{-1}=axa^{-1}aya^{-1}=\gamma_a(x)\gamma_a(y)$.
			Since $\gamma_a(xy)=\gamma_a(x)\gamma_a(y)$, $\gamma_a$ is a homomorphism.
			\par
			Suppose $\gamma_a(x)=\gamma_a(y)$. Then
			\begin{align*}
				axa^{-1}&=aya^{-1}\\
				axa^{-1}a&=aya^{-1}a\\
				a^{-1}axa^{-1}a&=a^{-1}aya^{-1}a\\
				1x1&=1y1\\
				x&=y
			\end{align*}
			$\gamma_a$ is injective.
			\par
			The closure of groups requires that $a,y\in G\implies ay\in G$. Also $a\in G\implies a^{-1}\in G$. Closure again gives that $aya^{-1}\in G$. We can see that, for any $y\in G$, we have a corresponding $aya^{-1}\in G$.
			(\dots\textit{ insufficient argument }\dots) still need to show that $\forall y\in G,\exists x\in G\text{ such that }\gamma_a(x)=y$.
			\par
			Since $\gamma_a$ is a bijective homomorphism, it is, by definition, an isomorphism.
		\end{proof}

		\item If $a,b\in G$, prove that $\gamma_a\gamma_b=\gamma_{ab}$.
		\bigskip
		\begin{proof}
			Let $x\in G$. $\gamma_{ab}(x)=(ab)x(ab)^{-1}=abxb^{-1}a^{-1}$. This is the same as $\gamma_a\gamma_b$, the composition of functions $\gamma_a$ and $\gamma_b$: $\gamma_a\circ \gamma_b(x)=a(bxb^{-1})a^{-1}$.
		\end{proof}
	\end{enumerate}
\end{exercise}

\begin{exercise}{1.50}{26}
	\quad
	\begin{enumerate}[(i.)]
		\item Prove that a group $G$ is abelian if and only if the function $f:G\to G$ given by $f(a)=a^{-1}$ is a homomorphism.
		\bigskip

		\item Let $f:G\to G$ be an isomorphism. If $f$ has no nontrivial fixed points (\textsc{iow}, $f(x)=x\implies x=e$) and if $f\circ f$ is the identity map, prove that $\forall x\in G$, $f(x)=x^{-1}$.
	\end{enumerate}
\end{exercise}

\begin{exercise}{2.2}{5}
	{\noindent}If $k$ is a field, show that $SL(n,k)$, the set of all $n\times n$ matrices over $k$ having determinant $1$, is a subgroup of $GL(n,k)$.
	\bigskip
\end{exercise}

\newpage

\begin{exercise}{2.9i}{5}
	{\noindent}Prove that $S_n$ can be generated by $(1\enspace 2), (1\enspace 3),\;\dotsc, (1\enspace n)$.
	\bigskip
\end{exercise}

\begin{exercise}{2.10}{5}
	{\noindent}Prove: If $G$ is a finite group and $K\le H\le G$, then $[G:K]=[G:H][H:K]$.
	\bigskip
\end{exercise}

\begin{exercise}{2.14}{5}
	{\noindent}Prove: If $a\in G$ has finite order and $f:G\to H$ is a homomorphism, then the order of $f(a)$ divides the order of $a$.
	\bigskip
\end{exercise}

\begin{exercise}{2.16}{5}
	{\noindent}Prove: If $H\le G$ has index $2$, then $\forall a\in G$, $a^2\in H$.
	\bigskip
\end{exercise}

\end{document}
