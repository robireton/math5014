\documentclass[12pt]{article}
\usepackage[top=1.0in, bottom=0.75in, left=0.75in, right=2.5in]{geometry}
\usepackage{fancyhdr, textpos, enumerate}
\usepackage{amsmath, amssymb, amsthm}
\usepackage{ebgaramond-maths, plex-mono}
\usepackage[makeroom]{cancel}

\pagestyle{fancy}
\fancyhf{}
\setlength\headheight{28pt}
\setlength\headwidth{7in}
\setlength{\TPHorizModule}{1pt}
\setlength{\TPVertModule}{1pt}
\lhead{\textsc{group theory\\spring 2018}}
\rhead{\textsc{homework bundle \#3\\page \thepage}}
\chead{\textsc{rob\\ireton}}

\newcommand{\zee}{\mathbb{Z}}

\newenvironment{exercise}[2]{\begin{textblock}{32}[1,0](0,#2)\noindent#1\end{textblock}}{\vspace{1in}}

\begin{document}
\begin{exercise}{1.47}{8}
	{\noindent}Let $G$ be a group and fix $a\in G$. Consider the function $\gamma_a:G\to G$ given by $\gamma_a(x)=axa^{-1}$. (This is called conjugation by $a$.)
	\begin{enumerate}[(i.)]
		\item Prove that $\gamma_a$ is an isomorphism.
		\bigskip
		\begin{proof}
			Let $x,y\in G$.
			\par
			$\gamma_a(xy)=axya^{-1}=ax1ya^{-1}=axa^{-1}aya^{-1}=\gamma_a(x)\gamma_a(y)$.
			Since $\gamma_a(xy)=\gamma_a(x)\gamma_a(y)$, $\gamma_a$ is a homomorphism.
			\par
			Suppose $\gamma_a(x)=\gamma_a(y)$. Then
			\begin{align*}
				axa^{-1}&=aya^{-1}\\
				axa^{-1}a&=aya^{-1}a\\
				a^{-1}axa^{-1}a&=a^{-1}aya^{-1}a\\
				1x1&=1y1\\
				x&=y
			\end{align*}
			$\gamma_a$ is injective.
			\par
			The closure of groups requires that $a,y\in G\implies ay\in G$. Also $a\in G\implies a^{-1}\in G$. Closure again gives that $aya^{-1}\in G$. We can see that, for any $y\in G$, we have a corresponding $aya^{-1}\in G$.
			(\dots\textit{ insufficient argument }\dots) still need to show that $\forall y\in G,\exists x\in G\text{ such that }\gamma_a(x)=y$.
			\par
			Since $\gamma_a$ is a bijective homomorphism, it is, by definition, an isomorphism.
		\end{proof}

		\item If $a,b\in G$, prove that $\gamma_a\gamma_b=\gamma_{ab}$.
		\bigskip
		\begin{proof}
			Let $x\in G$. $\gamma_{ab}(x)=(ab)x(ab)^{-1}=abxb^{-1}a^{-1}$. This is the same as $\gamma_a\gamma_b$, the composition of functions $\gamma_a$ and $\gamma_b$: $\gamma_a\circ \gamma_b(x)=a(bxb^{-1})a^{-1}$.
		\end{proof}
	\end{enumerate}
\end{exercise}

\begin{exercise}{1.50}{26}
	\quad
	\begin{enumerate}[(i.)]
		\item Prove that a group $G$ is abelian if and only if the function $f:G\to G$ given by $f(a)=a^{-1}$ is a homomorphism.
		\bigskip

		\item Let $f:G\to G$ be an isomorphism. If $f$ has no nontrivial fixed points (\textsc{iow}, $f(x)=x\implies x=e$) and if $f\circ f$ is the identity map, prove that $\forall x\in G$, $f(x)=x^{-1}$.
	\end{enumerate}
\end{exercise}

\newpage

\begin{exercise}{2.2}{5}
	{\noindent}If $k$ is a field, show that $\text{SL}(n,k)$, the set of all $n\times n$ matrices over $k$ having determinant $1$, is a subgroup of $\text{GL}(n,k)$.
	\bigskip
	\begin{proof}
		Let $k$ be a field and let $n\in\mathbb{N}$. $\text{GL}(n,k)$ is the group of $n\times n$ matrices with elements from $k$ having a nonzero determinant. Its identity element is the identity matrix $I_n$ and its operation is matrix multiplication.
		$\text{SL}(n,k)$ is the set of $n\times n$ matrices with elements from $k$ having a determinant equal to $1$. Observe that $\text{SL}(n,k)$ is a subset of $\text{GL}(n,k)$, specifically, all the elements of $\text{GL}(n,k)$ with determinant equal to $1$.
		We know from Linear Algebra that $\det(I_n)=1$, so $I_n\in \text{SL}(n,k)$.\\
		\\
		Let $s,t\in\text{SL}(n,k)$; then $\det(s)=1$ and $\det(t)=1$. $\det(t^{-1})=1/\det(t)=1$, so $t^{-1}\in\text{SL}(n,k)$.
		$\det(st^{-1})=\det(s)\det(t^{-1})=1$, so $st^{-1}\in\text{SL}(n,k)$.\\
		\\
		Since $\text{SL}(n,k)$ is a subset of $\text{GL}(n,k)$, $1\in \text{SL}(n,k)$ and $s, t\in \text{SL}(n,k)$ imply $st^{-1}\in\text{SL}(n,k)$, Theorem 2.2 tells us that $\text{SL}(n,k)$ is a subgroup of $\text{GL}(n,k)$.
	\end{proof}
\end{exercise}

\begin{exercise}{2.9i}{5}
	{\noindent}Prove that $S_n$ can be generated by $(1\enspace 2), (1\enspace 3),\;\dotsc, (1\enspace n)$.
	\bigskip
	\begin{proof}
		From Theorem 1.1, we know that every element of $S_n$ is either a cycle or a product of disjoint cycles.
		Each of those cycles has length 1, length 2, or length greater than 2.\\
		\\
		For cycles with length greater than 2, Theorem 1.3 shows how those cycles may be expressed as products from the set above. [$(1\enspace 2), (1\enspace 3), $ \textit{etc.}]\\
		\\
		For transpositions, observe that $(i\enspace j)$ may be expressed as $(1\enspace i)(1\enspace j)(1\enspace i)$.\\
		\\
		Fixed points $(i)$ are all equal to $1$ and do not affect the product.\\
		\\
		So, for each cycle in the product that makes up each element of $S_n$, we have a way to express that cycle as a product of elements of the set above, giving us a way to generate all of $S_n$.
	\end{proof}
\end{exercise}

\newpage

\begin{exercise}{2.10}{5}
	{\noindent}Prove: If $G$ is a finite group and $K\le H\le G$, then $[G:K]=[G:H][H:K]$.
	\bigskip
	\begin{proof}
		From Lagrange's Theorem (2.11) we have
		\begin{align*}
			[G:H]&=|G|/|H|\\
			[H:K]&=|H|/|K|\\
			[G:K]&=|G|/|K|
		\end{align*}
		All of these indexes and orders are natural numbers, so we can rearrange to get $|G|=[G:H]|H|$ and $|H|=[H:K]|K|$. Substitution gives
		\begin{align*}
			[G:K]&=|G|/|K|\\
			&=[G:H]|H|/|K|\\
			&=[G:H][H:K]|K|/|K|\\
			&=[G:H][H:K]
		\end{align*}
	\end{proof}
\end{exercise}

\begin{exercise}{2.14}{5}
	{\noindent}Prove: If $a\in G$ has finite order and $f:G\to H$ is a homomorphism, then the order of $f(a)$ divides the order of $a$.
	\bigskip
\end{exercise}

\begin{exercise}{2.16}{5}
	{\noindent}Prove: If $H\le G$ has index $2$, then $\forall a\in G$, $a^2\in H$.
	\bigskip
\end{exercise}

\end{document}
