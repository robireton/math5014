\documentclass[12pt]{article}
\usepackage[top=1.0in, bottom=0.75in, left=0.75in, right=2.5in]{geometry}
\usepackage{fancyhdr, textpos, faktor, enumerate}
\usepackage{amsmath, amssymb, amsthm}
\usepackage{ebgaramond-maths, plex-mono}

\pagestyle{fancy}
\fancyhf{}
\setlength\headheight{28pt}
\setlength\headwidth{7in}
\setlength{\TPHorizModule}{1pt}
\setlength{\TPVertModule}{1pt}
\lhead{\textsc{group theory\\spring 2018}}
\rhead{\textsc{homework bundle \#10\\page \thepage}}
\chead{\textsc{rob\\ireton}}

\newenvironment{exercise}[2]{\begin{textblock}{32}[1,0](0,#2)\noindent#1\end{textblock}}{\vspace{1in}}

\begin{document}
  % \begin{exercise}{3.42}{6}
  %   {\noindent}Prove: If $H\le G$, then $G$ acts transitively on the set of all left cosets of $H$, and $G$ acts transitively on the set of conjugates of $H$.
  % \end{exercise}

  \begin{exercise}{4.2}{6}
    {\noindent}Prove: If $|G|=p^n$, where $p$ is prime, and if $0\le k\le n$, then $G$ contains a normal subgroup of order $p^k$.

    \begin{proof}
      $|G|=p^n\implies G$ is a $p$-group by corollary 4.3.
      By theorem 4.3 (Cauchy), $G$ contains an element, call it $g_1$, of order $p$.
      The subgroup generated by that element is normal in $G$ and has order $p$.
      $[G:\langle g_1 \rangle]=p^{n-1}$.
    \end{proof}
  \end{exercise}

  % \begin{exercise}{4.11}{6}
  %   {\noindent}Let $P\le G$ be a Sylow subgroup. Prove: if $N_G(P)\le H\le G$, then $H$ is equal to its own normalizer (that is, $H=N_G(H)$). Assume $G$ is finite.
  % \end{exercise}
  %
  % \begin{exercise}{4.30}{6}
  %   {\noindent}For every divisor $d$ of 24, prove that there is a subgroup of $S_4$ of order $d$. Moreover, if $d\ne 4$, then any two subgroups of order $d$ are isomorphic.
  % \end{exercise}

  % \begin{proof}
  %   The divisors of $24$ are $1,2,3,4,6,8,12,24$.\\
  %   \\
  %   \begin{tabular}{l r r r}
  %   	$d=1$  & $\{e\}\leq S_4$        &  $|\{e\}|=$        & $1$ \\
  %   	$d=2$  & $\mathbb{Z}_2\leq S_4$ &  $|\mathbb{Z}_2|=$ & $2$ \\
  %   	$d=3$  & $\mathbb{Z}_3\leq S_4$ &  $|\mathbb{Z}_3|=$ & $3$ \\
  %     $d=4$  & $\mathbf{V}\leq S_4$   &  $|\mathbf{V}|=$   & $4$ \\
  %   	$d=6$  & $? \leq S_4$           &  $|?|=$            & $6$ \\
  %   	$d=8$  & $? \leq S_4$           &  $|?|=$            & $8$ \\
  %   	$d=12$ & $A_4\leq S_4$          &  $|A_4|=$          & $12$\\
  %   	$d=24$ & $S_4\leq S_4$          &  $|S_4|=$          & $24$\\
  %   \end{tabular}
  %   \\
  %   \\
  % \end{proof}
  %
  % \begin{exercise}{4.36}{6}
  %   {\noindent}Prove that there is no simple nonabelian group of order less than 60.
  % \end{exercise}
  %
  % \newpage

  % \begin{exercise}{5.5}{6}
  %   {\noindent}Let $f(x)=x^n-a\,\in F[x]$, let $E$ be the splitting field of $f(x)$ over $F$, and let $\alpha\in E$ be an $n^{th}$ root of $a$. Prove that there are subfields
  %   \[F=K_0\subset K_1\subset \dotsb \subset K_t=F(\alpha)\] with
  %   \begin{itemize}
  %   	\item $K_{i+1}=K_I(\beta_{i+1})$ for some $\beta_{i+1}\in F(\alpha)$,
  %   	\item For each $i$, $\exists p(i)$ prime such that $\beta_{i+1}^{p(i)}\in K_i$.
  %   \end{itemize}
  % \end{exercise}

  \begin{exercise}{5.8}{6}
    {\noindent}Prove: Every finite group has a composition series.

    \begin{proof}
      Let $G$ be a finite group.
      If $G$ is simple, then its composition series is just $1\vartriangleleft G$.
      If $G$ is not simple, then it has at least one non-trivial proper normal subgroup.
      One of these, call it $G_1$, must be maximal, so we have $1\vartriangleleft G_1\vartriangleleft G$.\\
      Now, either $G_1$ is simple or not\dots.\\
      Iterate as above until $G_i$ is simple. $G$ is finite, so we can always construct a composition series for it in this way.
    \end{proof}
  \end{exercise}

  % \begin{exercise}{5.10}{6}
  %   {\noindent}If $G$ is a finite group having a normal series with factor groups $H_0,H_1\dotsc, H_n$, then $|G|=\prod |H_i|$.
  % \end{exercise}
  %
  % \begin{exercise}{5.13}{6}
  %   {\noindent}Assume that $G=H_1\times \dotsb \times H_n= K_1\times \dotsb \times K_m$, where each $K_i$ and $H_J$ is simple. Prove that $m=n$ and that there is a permutation $\pi\in S_n$ such that for all $i$, $H_i\cong K_{\pi(i)}$.
  %   [Hint: Construct composition series for $G$.]
  % \end{exercise}

\end{document}
