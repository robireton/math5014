\documentclass[12pt]{article}
\usepackage[top=1.0in, bottom=0.75in, left=0.75in, right=2.5in]{geometry}
\usepackage{fancyhdr, textpos, faktor}
\usepackage{amsmath, amssymb, amsthm}
\usepackage{ebgaramond-maths, plex-mono}

\pagestyle{fancy}
\fancyhf{}
\setlength\headheight{28pt}
\setlength\headwidth{7in}
\setlength{\TPHorizModule}{1pt}
\setlength{\TPVertModule}{1pt}
\lhead{\textsc{group theory\\spring 2018}}
\rhead{\textsc{homework bundle \#4\\page \thepage}}
\chead{\textsc{rob\\ireton}}

\newcommand{\zee}{\mathbb{Z}}
\newcommand{\normal}{\vartriangleleft}

\newenvironment{exercise}[2]{\begin{textblock}{32}[1,0](0,#2)\noindent#1\end{textblock}}{\vspace{1in}}

\begin{document}

\begin{exercise}{2.23}{6}
	{\noindent}If $\gcd(r,s)=1$, then $s^{\varphi(r)}\equiv 1 \mod r$.
	[\textit{Hint}. The order of the group of units $U(\zee_n)$ is $\varphi(n)$.]
	\bigskip

	$r$ and $s$ are relatively prime.\\
	$\varphi(r)$ is the number of positive integers that are relatively prime to $r$.\\
	Need to show that $r$ divides $s^{\varphi(r)}-1$.\\
	\dots
\end{exercise}

\begin{exercise}{2.26}{6}
	{\noindent}Let $\{S_i;\; i\in I\}$ be a family of subgroups of a group $G$, let $\{S_it_i;\; i\in I\}$ be a family of right cosets, and let $D=\bigcap_{i\in I}S_i$.  Prove that either $\bigcap_{i\in I}S_it_i=\emptyset$ or $\bigcap_{i\in I}S_it_i=Dg$ for some $g\in G$.
	\bigskip

	\dots
\end{exercise}

\begin{exercise}{2.30}{6}
	{\noindent}If $S\le G$ and $[G:S]=2$, then $S\normal G$.
	\bigskip

	There are two right cosets of $S$ in $G$: $\{Sg_1, Sg_2\}$, with $g_1, g_2$ representatives of their respective right cosets.\\
	For $S\vartriangleleft G$ to be true, $\forall s\in S, \forall g\in G, gsg^{-1}\in S$.\\
	\dots

\end{exercise}

\begin{exercise}{2.36}{6}
	{\noindent}Prove that $A_n\normal S_n$ for every $n\ge 2$.
	\bigskip

	$A_n$, the set of all even permutations in $S_n$ is a group with order $n!/2$. (exercise 2.1)\\
	Any particular $S_n$ is finite, so $|G|/|S|=[G:S]=2$ (Lagrange).\\
	$[G:S]=2$, so this should follow, more or less, from exercise 2.30, which I haven't managed to prove, yet.\\
	\dots
\end{exercise}

\newpage

\begin{exercise}{2.40}{6}
	{\noindent}Let $H\normal G$, let $v:G\to \faktor{G}{H}$ be the natural map, and let $X\subset G$ be a subset such that $v(X)$ generates $\faktor{G}{H}$. Prove that $G=\langle H\cup X\rangle$.
	\bigskip

	$\forall h\in H, \forall g\in G, ghg^{-1}\in H$.\\
	$\nu: g\mapsto Hg$.\\
	$\faktor{G}{H}=\langle\nu(X)\rangle=\langle HX\rangle$\\
	\dots
\end{exercise}

\begin{exercise}{2.41}{6}
	{\noindent}Let $G$ be a finite group of odd order, and let $x$ be the product of all the elements of $G$ (in some order). Prove that $x\in G'$.
	\bigskip

	$G=\{e, g_1, g_2, \dots, g_n\}$\\
	$|G|$ is odd, so $n$ is even, so non-identity elements of $G$ can be `paired up' if needed.\\
	Maybe those `pairs' are elements and their inverses?\\
	$x=product(shuffle(elements(G)))$. $G$ is closed so $x\in G$.\\
	$G'=\langle\{aba^{-1}b^{-1}: a,b\in G\}\rangle$.\\
	\dots
\end{exercise}

\end{document}
