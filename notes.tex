\documentclass[12pt]{report}
\usepackage[top=0.75in, bottom=0.75in, left=1in, right=1in]{geometry}
\usepackage{amsmath, amssymb, amsthm}
\title{Group Theory Notes}

\newtheorem{theorem}{Theorem}[chapter]
\newtheorem{corollary}[theorem]{Corollary}
\newtheorem{lemma}[theorem]{Lemma}
\theoremstyle{definition}
\newtheorem*{definition}{Definition}
\newtheorem*{notation}{Notation}

\newcommand{\term}[1]{\textbf{\textit{#1}}}
\newcommand{\gen}[1]{{\langle}#1{\rangle}}

\begin{document}
\chapter{Groups and Homomorphisms}
\chapter{The Isomorphism Theorems}
\section{Subgroups}
\begin{definition}
	A nonempty subset $S$ of a group $G$ is a \term{subgroup} of $G$ if $s\in G$ implies $s^{-1}\in G$ and $s, t\in G$ imply $st\in G$.
\end{definition}

\begin{theorem}
	If $S\leq G$ (i.e., if $S$ is a subgroup of G), then $S$ is a group in its own right.
\end{theorem}

\begin{theorem}
	A subset $S$ of a group $G$ is a subgroup if and only if $1\in S$ and $s, t\in S$ imply $st^{-1}\in S$.
\end{theorem}

\begin{definition}
	If $G$ is a group and $a\in G$, then the \term{cyclic subgroup generated by a}, denoted by $\gen{a}$, is the set of all the powers of $a$.
	A group $G$ is called \term{cyclic} if there is $a\in G$ with $G=\gen{a}$; that is, $G$ consists of all the powers of $a$.
\end{definition}

\begin{definition}
	If $G$ is a group and $a\in G$, then the \term{order} of $a$ is $|\gen{a}|$, the number of elements in $\gen{a}$.
\end{definition}

\begin{theorem}
	If $G$ is a group and $a\in G$ has finite order $m$, then $m$ is the smallest positive integer such that $a^m=1$.
\end{theorem}

\begin{corollary}
	If $G$ is a finite group, then a nonempty subset $S$ of $G$ is a subgroup if and only if $s, t\in S$ imply $st \in S$.
\end{corollary}

\begin{definition}
	Let $f:G\to H$ be a homomorphism, then $\term{kernel}\ f = \{ a\in G: f(a)=1\}$ and $\term{image}\ f = \{ h\in H: h=f(a) \text{ for some } a\in G \}$.
\end{definition}

\begin{notation}
	We usually write \term{ker f} instead of kernel $f$ and \term{im f} instead of image $f$.
\end{notation}

\begin{theorem}
	The intersection of any family of subgroups of a group $G$ is again a subgroup of $G$.
\end{theorem}

\begin{corollary}
	If $X$ is a subset of a group $G$, then there is a \textbf{smallest} subgroup $H$ of $G$ containing $X$; that is, if $X\subset S$ and $S\leq G$, then $H\leq S$.
\end{corollary}

\begin{definition}
	If $X$ is a subset of a group $G$, then the smallest subgroup of $G$ containing $X$, denoted by $\gen{X}$, is called the \term{subgroup generated by X}.
	One also says that $X$ \term{generates} $\gen{X}$.
\end{definition}

\begin{notation}
	If $H$ and $K$ are subgroups of $G$, then the subgroup $\gen{H\cup K}$ is denoted by $H\vee K$.
\end{notation}

\begin{definition}
	If $X$ is a nonempty subset of a group $G$, then a \term{word} on X is an element $w\in G$ of the form
	\[ w = x_1^{e_1} x_2^{e_2} \cdots x_n^{e_n}, \]
	where $x_i\in X$, $e_i=\pm1$, and $n\geq1$.
\end{definition}

\begin{theorem}
	Let $X$ be a subset of a group $G$. If $X=\varnothing$, then $\gen{X}=1$; if $X$ is nonempty, then $\gen{X}$ is the set of all the words on $X$.
\end{theorem}

\section{Lagrange's Theorem}
\begin{definition}
	If $S$ is a subgroup of $G$ and if $t\in G$, then a \term{right coset} of $S$ in $G$ is the subset of $G$
	\[ St = \{ st: s\in S \} \]
	(a \term{left coset} is $tS = \{ ts: s\in S \}$). One calls $t$ a \term{representative} of $St$ (and also of $tS$).
\end{definition}

\begin{lemma}
	If $S\leq G$, then $Sa=Sb$ if and only if $ab^{-1}\in S$ ($aS=bS$ if and only if $b^{-1}a\in S$).
\end{lemma}

\begin{theorem}
	If $S\leq G$, then any two right (or any two left) cosets of $S$ in $G$ are either identical or disjoint.
\end{theorem}

\begin{theorem}
	If $S\leq G$, then the number of right cosets of $S$ in $G$ is equal to the number of left cosets of $S$ in $G$.
\end{theorem}

\begin{definition}
	If $S\leq G$, then the \term{index} of $S$ in $G$, denoted by $[G:S]$, is the number of right cosets of $S$ in $G$.
\end{definition}

\begin{definition}
	If $G$ is a group, then the \term{order} of $G$, denoted by $|G|$, is the number of elements in $G$.
\end{definition}

\begin{theorem}[Lagrange]
	If $G$ is a finite group and $S\leq G$, then $|S|$ divides $|G|$ and $[G:S] = |G|/|S|$.
\end{theorem}

\begin{corollary}
	If $G$ is a finite group and $a\in G$, then the order of $a$ divides $|G|$.
\end{corollary}

\begin{definition}
	A group $G$ has \term{exponent} n if $x^n=1$ for all $x\in G$.
\end{definition}

\begin{corollary}
	If $p$ is a prime and $|G|=p$, then $G$ is a cyclic group.
\end{corollary}

\begin{corollary}[Fermat]
	If $p$ is a prime and $a$ is an integer, then $a^p\equiv a\mod p$.
\end{corollary}

\begin{definition}
	The \term{Euler \varphi-function} is defined as follows:
	\[ \varphi(1)=1;\quad\text{ if }n>1,\text{ then }\varphi(n)=|\{k:1\leq k<n\text{ and }(k, n)=1\}|. \]
\end{definition}

\section{Cyclic Groups}

\begin{lemma}
	If $G$ is a cyclic group of order $n$, then there exists a unique subgroup of order $d$ for every divisor $d$ of $n$.
\end{lemma}

\begin{theorem}
	If $n$ is a positive integer, then
	\[ n=\sum_{d\mid n} \varphi(d), \]
	where the sum is over all divisors $d$ of $n$ with $1\leq d\leq n$.
\end{theorem}

\begin{theorem}
	A group $G$ of order $n$ is cyclic if and only if, for each divisor $d$ of $n$, there is at most one cyclic subgroup of $G$ having order $d$.
\end{theorem}

\begin{theorem}
	\quad
	\renewcommand{\labelenumi}{\textnormal{(\roman{enumi})}}
	\begin{enumerate}
		\item If $F$ is a field and if $G$ is a finite subgroup of $F^\times$, the multiplicative group of nonzero elements of $F$, then $G$ is cyclic.
		\item If $F$ is a finite field, then its multiplicative group $F^\times$ is cyclic.
	\end{enumerate}
\end{theorem}

\begin{theorem}
	Let $p$ be a prime. A group $G$ of order $p^n$ is cyclic if and only if it is an abelian group having a unique subgroup of order $p$.
\end{theorem}
\end{document}
