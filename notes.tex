\documentclass[12pt]{report}
\usepackage[top=0.75in, bottom=1in, left=1in, right=1in]{geometry}
\usepackage{amsmath, amssymb, amsthm, faktor}
\usepackage{mathspec} %loads fontspec as well
\setallmainfonts(Digits,Latin,Greek){Alegreya}

\title{Group Theory Notes}

\newtheorem{theorem}{Theorem}[chapter]
\newtheorem{corollary}[theorem]{Corollary}
\newtheorem{lemma}[theorem]{Lemma}
\theoremstyle{definition}
\newtheorem*{definition}{Definition}
\newtheorem*{notation}{Notation}
\newtheorem*{remark}{Remark}

\newcommand{\term}[1]{\textbf{\textit{#1}}}
\newcommand{\gen}[1]{{\langle}#1{\rangle}}

\renewcommand{\labelenumi}{\textnormal{(\roman{enumi})}}

\begin{document}
\chapter{Groups and Homomorphisms}
\section{Permutations}
\begin{definition}
	If $X$ is a nonempty set, a \term{permutation} of $X$ is a bijection $\alpha:X\to X$. We denote the set of all permutations of $X$ by $S_X$.
\end{definition}

\section{Cycles}
\begin{definition}
	If $x\in X$ and $\alpha\in S_X$, then $\alpha$ \term{fixes} $x$ if $\alpha(x)=x$ and $\alpha$ \term{moves} $x$ if $\alpha(x)\neq x$.
\end{definition}

\begin{definition}
	Let $i_1, i_2, \dots, i_r$ be distinct integers between $1$ and $n$. If $\alpha\in S_n$ fixes the remaining $n-r$ integers and if
	\[ \alpha(i_1)=i_2,\;\alpha(i_2)=i_3,\;\dots,\;\alpha(i_{r-1})=i_r,\;\alpha(i_r)=i_1, \]
	then $\alpha$ is an \term{r-cycle}; one also says that $\alpha$ is a cycle of \term{length} r. Denote $\alpha$ by $(i_1\enspace i_2\enspace \cdots \enspace i_r)$.
	\smallskip

	Every 1-cycle fixes every element of $X$, and so all 1-cycles are equal to the identity. A 2-cycle, which merely interchanges a pair of elements, is called a \term{transposition}.
\end{definition}

\begin{definition}
	Two permutations $\alpha, \beta\in S_X$ are \term{disjoint} if every $x$ moved by one is fixed by the other.
	In symbols, if $\alpha(x)\neq x$, then $\beta(x)=x$ and if $\beta(y)\neq y$, then $\alpha(y)=y$
	(of course it is possible that there is $z\in X$ with $\alpha(z)=z=\beta(z)$).
	A family of permutations $\alpha_1,\alpha_2,\dots,\alpha_m$ is \term{disjoint} if each pair of them is disjoint.
\end{definition}

\section{Factorization into Disjoint Cycles}
\begin{theorem}
	Every permutation $\alpha\in S_n$ is either a cycle or a product of disjoint cycles.
\end{theorem}

\begin{definition}
	A \term{complete factorization} of a permutation $\alpha$ is a factorization of $\alpha$ as a product of disjoint cycles which contains one 1-cycle $(i)$ for every $i$ fixed by $\alpha$.
	\smallskip

	In a complete factorization of a permutation $\alpha$, every $i$ between $1$ and $n$ occurs in exactly one of the cycles.
\end{definition}

\begin{theorem}
	Let $\alpha\in S_n$ and let $\alpha=\beta_1\dots\beta_t$ be a complete factorization into disjoint cycles. This factorization is unique except for the order in which the factors occur.
\end{theorem}

\section{Even and Odd Permutations}
\begin{theorem}
	Every permutation $\alpha\in S_n$ is a product of transpositions. For each cycle of length $r>1$:
	\[ (i_1\enspace i_2\enspace\dots\enspace i_r)=(i_1\enspace i_r)(i_1\enspace i_{r-1})\dots(i_1\enspace i_2)\]
\end{theorem}

\begin{definition}
	A permutation $\alpha\in S$ is \term{even} if it is a product of an even number of transpositions; otherwise, $\alpha$ is \term{odd}.
\end{definition}

\begin{lemma}
	If $k, l\geq0$, then
	\[ (a\enspace b)(a\enspace c_1\enspace\dots\enspace c_k\enspace b\enspace d_1\enspace\dots\enspace d_l) = (a\enspace c_1\enspace\dots\enspace c_k)(b\enspace d_1\enspace\dots\enspace d_l) \]
	and
	\[ (a\enspace b)(a\enspace c_1\enspace\dots\enspace c_k)(b\enspace d_1\enspace\dots\enspace d_l) = (a\enspace c_1\enspace\dots\enspace c_k\enspace b\enspace d_1\enspace\dots\enspace d_l) \]
\end{lemma}

\begin{definition}
	If $\alpha\in S_n$ and $\alpha=\beta_1\dots\beta_t$ is a complete factorization into disjoint cycles, then \term{signum} $\alpha$ is defined by
	\[ \text{sgn}(\alpha) = (-1)^{n-t}. \]
\end{definition}

\begin{lemma}
	If $\beta\in S_n$ and $\tau$ is a transposition, then
	\[ \textnormal{sgn}(\tau\beta) = -\textnormal{sgn}(\beta). \]
\end{lemma}

\begin{lemma}
	For all $\alpha, \beta\in S_n$,
	\[ \textnormal{sgn}(\alpha\beta) = \textnormal{sgn}(\alpha)\,\textnormal{sgn}(\beta). \]
\end{lemma}

\begin{theorem}
	\quad
	\begin{enumerate}
		\item A permutation $\alpha\in S_n$ is even if and only if $\textnormal{sgn}(\alpha)=1$.
		\item A permutation $\alpha\in S_n$ is odd if and only if it is a product of an odd number of transpositions.
	\end{enumerate}
\end{theorem}

\section{Semigroups}
\begin{definition}
	A (binary) \term{operation} on a nonempty set $G$ is a function $\mu:G\times G\to G$.
\end{definition}

\begin{definition}
	An operation $\ast$ on a set $G$ is \term{associative} if \[(a\ast b)\ast c=a\ast(b\ast c) \] for every $a, b, c\in G$.
\end{definition}

\begin{definition}
	An expression $a_1\ast a_2\ast\cdots\ast a_n$ \term{needs no parentheses} if, no matter what choices of multiplications of adjacent factors are made, the resulting elements of $G$ are all equal.
\end{definition}

\begin{theorem}[Generalized Associativity]
	If $\ast$ is an associative operation on a set $G$, then every expression $a_1\ast a_2\ast\cdots\ast a_n$ needs no parentheses.
\end{theorem}

\begin{definition}
	A \term{semigroup} $(G,\ast)$ is a nonempty set $G$ equipped with an associative operation $\ast$.
\end{definition}

\begin{definition}
	Let $G$ be a semigroup and let $a\in G$. Define $a^1=a$ and, for $n\geq1$, define $a^{n+1}=a\ast a^n$.
\end{definition}

\begin{corollary}
	Let $G$ be a semigroup, let $a\in G$, and let $m$ and $n$ be positive integers. Then $a^m\ast a^n=a^{m+n}=a^n\ast a^m$ and $(a^m)^n=a^{mn}=(a^n)^m$.
\end{corollary}


\section{Groups}
\begin{definition}
	A \term{group} is a semigroup $G$ containing an element $e$ such that:
	\begin{enumerate}
		\item $e\ast a=a=a\ast e$ for all $a\in G$;
		\item for every $a\in G$, there is an element $b\in G$ with $a\ast b=e=b\ast a$.
	\end{enumerate}
	\par
	$S_X$ is a group with composition as operation. it is called the \term{symmetric group} on $X$. When $X=\{1, 2, \dots, n\}$,
	then $S_X$ is denoted by $S_n$ and it is called the \term{symmetric group on n letters}.
\end{definition}

\begin{definition}
	A pair of elements $a$ and $b$ in a semigroup \term{commutes} if $a\ast b=b\ast a$. A group (or semigroup) is \term{abelian} if every pair of its elements commutes.
\end{definition}

\begin{theorem}
	If $G$ is a group, then there is a unique element $e$ with $e\ast a=a=a\ast e$ for all $a\in G$.
	Moreover, for each $a\in G$, there is a unique $b\in G$ with $a\ast b=e=b\ast a$.
\end{theorem}

\begin{remark}
	As a result of the uniqueness assertions of the theorem, we may now give names to $e$ and $b$. We call $e$ the \term{identity} of $G$ and,
	if $a\ast b=e=b\ast a$, then we call $b$ the \term{inverse} of $a$ abd denote it by $a^{-1}$.
\end{remark}

\begin{corollary}
	If $g$ is a group and $a\in G$, then $(a^{-1})^{-1}=a$.
\end{corollary}

\begin{definition}
	If $G$ is a group and $a\in G$, define the \term{powers} of $a$ as follows: if $n$ is a positive integer, then $a^n$ is defined as in any semigroup; define $a^0=e$; define $a^{-n}=(a^{-1})^n$.
\end{definition}

\begin{theorem}
	If $G$ is a semigroup with an element $e$ such that:
	\begin{enumerate}
		\item $e\ast a=a$ for all $a\in G$; and
		\item for each $a\in G$ there is an element $b\in G$ with $b\ast a=e$, then $G$ is a group.
	\end{enumerate}
\end{theorem}

\begin{remark}
	The four permutations $\{1,\;(1\enspace2)(3\enspace4),\;(1\enspace3)(2\enspace4),\;(1\enspace4)(2\enspace3) \}$
	form a group $\mathbf{V}$ called the \term{Klein 4-group}.
\end{remark}

\section{Homomorphisms}
\begin{definition}
	Let $(G,\ast)$ and $(H,\circ)$ be groups. A function $f:G\to H$ is a \term{homomorphism} if, for all $a,b\in G$, \[f(a\ast b)=f(a)\circ f(b).\]
	An \term{isomorphism} is a homomorphism that is also a bijection. We say that $G$ is \term{isomorphic} to $H$, denoted by $G\cong H$, if there exists an isomorphism $f:G\to H$.
\end{definition}

\begin{theorem}
	Let $f:(G,\ast)\to(G',\circ)$ be a homomorphism.
	\begin{enumerate}
		\item $f(e)=e'$, where $e'$ is the identity in $G'$.
		\item if $a\in G$, then $f(a^{-1})=f(a)^{-1}$.
		\item if $a\in G$ and $n\in\mathbb{Z}$, then $f(a^n)=f(a)^n$.
	\end{enumerate}
\end{theorem}



\chapter{The Isomorphism Theorems}
\begin{notation}
	We now drop the $\ast$ notation of the operation in a group. Henceforth, we shall write $ab$ instead of $a\ast b$, and we shall denote the identity element by $1$ instead of by $e$.
\end{notation}

\section{Subgroups}
\begin{definition}
	A nonempty subset $S$ of a group $G$ is a \term{subgroup} of $G$
	if $s\in S$ implies $s^{-1}\in S$ and $s, t\in S$ imply $st\in S$.
\end{definition}

\begin{theorem}
	If $S\leq G$ (i.e., if $S$ is a subgroup of G), then $S$ is a group in its own right.
\end{theorem}

\begin{theorem}
	A subset $S$ of a group $G$ is a subgroup if and only if $1\in S$ and $s, t\in S$ imply $st^{-1}\in S$.
\end{theorem}

\begin{definition}
	If $G$ is a group and $a\in G$, then the \term{cyclic subgroup generated by a}, denoted by $\gen{a}$, is the set of all the powers of $a$.
	A group $G$ is called \term{cyclic} if there is $a\in G$ with $G=\gen{a}$; that is, $G$ consists of all the powers of $a$.
\end{definition}

\begin{definition}
	If $G$ is a group and $a\in G$, then the \term{order} of $a$ is $|\gen{a}|$, the number of elements in $\gen{a}$.
\end{definition}

\begin{theorem}
	If $G$ is a group and $a\in G$ has finite order $m$, then $m$ is the smallest positive integer such that $a^m=1$.
\end{theorem}

\begin{corollary}
	If $G$ is a finite group, then a nonempty subset $S$ of $G$ is a subgroup if and only if $s, t\in S$ imply $st \in S$.
\end{corollary}

\begin{definition}
	Let $f:G\to H$ be a homomorphism, then $\term{kernel}\ f = \{ a\in G: f(a)=1\}$ and $\term{image}\ f = \{ h\in H: h=f(a) \text{ for some } a\in G \}$.
\end{definition}

\begin{notation}
	We usually write \term{ker f} instead of kernel $f$ and \term{im f} instead of image $f$.
\end{notation}

\begin{theorem}
	The intersection of any family of subgroups of a group $G$ is again a subgroup of $G$.
\end{theorem}

\begin{corollary}
	If $X$ is a subset of a group $G$, then there is a \textbf{smallest} subgroup $H$ of $G$ containing $X$; that is, if $X\subset S$ and $S\leq G$, then $H\leq S$.
\end{corollary}

\begin{definition}
	If $X$ is a subset of a group $G$, then the smallest subgroup of $G$ containing $X$, denoted by $\gen{X}$, is called the \term{subgroup generated by X}.
	One also says that $X$ \term{generates} $\gen{X}$.
\end{definition}

\begin{notation}
	If $H$ and $K$ are subgroups of $G$, then the subgroup $\gen{H\cup K}$ is denoted by $H\vee K$.
\end{notation}

\begin{definition}
	If $X$ is a nonempty subset of a group $G$, then a \term{word} on X is an element $w\in G$ of the form
	\[ w = x_1^{e_1} x_2^{e_2} \cdots x_n^{e_n}, \]
	where $x_i\in X$, $e_i=\pm1$, and $n\geq1$.
\end{definition}

\begin{theorem}
	Let $X$ be a subset of a group $G$. If $X=\varnothing$, then $\gen{X}=1$; if $X$ is nonempty, then $\gen{X}$ is the set of all the words on $X$.
\end{theorem}

\section{Lagrange's Theorem}
\begin{definition}
	If $S$ is a subgroup of $G$ and if $t\in G$, then a \term{right coset} of $S$ in $G$ is the subset of $G$
	\[ St = \{ st: s\in S \} \]
	(a \term{left coset} is $tS = \{ ts: s\in S \}$). One calls $t$ a \term{representative} of $St$ (and also of $tS$).
\end{definition}

\begin{lemma}
	If $S\leq G$, then $Sa=Sb$ if and only if $ab^{-1}\in S$ ($aS=bS$ if and only if $b^{-1}a\in S$).
\end{lemma}

\begin{theorem}
	If $S\leq G$, then any two right (or any two left) cosets of $S$ in $G$ are either identical or disjoint.
\end{theorem}

\begin{theorem}
	If $S\leq G$, then the number of right cosets of $S$ in $G$ is equal to the number of left cosets of $S$ in $G$.
\end{theorem}

\begin{definition}
	If $S\leq G$, then the \term{index} of $S$ in $G$, denoted by $[G:S]$, is the number of right cosets of $S$ in $G$.
\end{definition}

\begin{definition}
	If $G$ is a group, then the \term{order} of $G$, denoted by $|G|$, is the number of elements in $G$.
\end{definition}

\begin{theorem}[Lagrange]
	If $G$ is a finite group and $S\leq G$, then $|S|$ divides $|G|$ and $[G:S] = |G|/|S|$.
\end{theorem}

\begin{corollary}
	If $G$ is a finite group and $a\in G$, then the order of $a$ divides $|G|$.
\end{corollary}

\begin{definition}
	A group $G$ has \term{exponent} n if $x^n=1$ for all $x\in G$.
\end{definition}

\begin{corollary}
	If $p$ is a prime and $|G|=p$, then $G$ is a cyclic group.
\end{corollary}

\begin{corollary}[Fermat]
	If $p$ is a prime and $a$ is an integer, then $a^p\equiv a\mod p$.
\end{corollary}


\section{Cyclic Groups}

\begin{definition}
	The \term{Euler $\varphi$-function}, also called \term{Euler's totient function} is defined as follows:
	\[ \varphi(1)=1;\quad\text{ if }n>1,\text{ then }\varphi(n)=|\{k:1\leq k<n\text{ and }\gcd(k, n)=1\}|. \]
	In other words, $\varphi(n)$ is the number of positive integers that are relatively prime to $n$.
\end{definition}

\begin{lemma}
	If $G$ is a cyclic group of order $n$, then there exists a unique subgroup of order $d$ for every divisor $d$ of $n$.
\end{lemma}

\begin{theorem}
	If $n$ is a positive integer, then
	\[ n=\sum_{d\mid n} \varphi(d), \]
	where the sum is over all divisors $d$ of $n$ with $1\leq d\leq n$.
\end{theorem}

\begin{theorem}
	A group $G$ of order $n$ is cyclic if and only if, for each divisor $d$ of $n$, there is at most one cyclic subgroup of $G$ having order $d$.
\end{theorem}

\begin{theorem}
	\quad
	\renewcommand{\labelenumi}{\textnormal{(\roman{enumi})}}
	\begin{enumerate}
		\item If $F$ is a field and if $G$ is a finite subgroup of $F^\times$, the multiplicative group of nonzero elements of $F$, then $G$ is cyclic.
		\item If $F$ is a finite field, then its multiplicative group $F^\times$ is cyclic.
	\end{enumerate}
\end{theorem}

\begin{theorem}
	Let $p$ be a prime. A group $G$ of order $p^n$ is cyclic if and only if it is an abelian group having a unique subgroup of order $p$.
\end{theorem}


\section{Normal Subgroups}

\begin{definition}
	If $S$ and $T$ are nonempty subsets of a group $G$, then \[ ST=\{st:s\in S\text{ and }T\in T\}.\]\par
	If $S\leq G$, $t\in G$, and $T=\{t\}$, then $ST$ is the right coset $St$.
	Notice that the family of all the nonempty subsets $G$ is a semigroup under this operation: if $S$, $T$, and $U$ are nonempty subsets of $G$,
	then $(ST)U=S(TU)$, for either side consists of all the elements of $G$ of the form $(st)u=s(tu)$ with $s\in S$, $t\in T$; and $u\in U$.
\end{definition}

\begin{theorem}[Product Formula]
	If $S$ and $T$ are subgroups of a finite group $G$, then \[|ST| |S\cap T|=|S| |T|. \]
	Note that the subset $ST$ need not be a subgroup.
\end{theorem}

\begin{definition}
	A subgroup $K\leq G$ is a \term{normal subgroup}, denoted by $K\vartriangleleft G$, if $gKg^{-1}=K$ for every $g\in G$.
\end{definition}

\begin{remark}
	A subgroup $K$ of $G$ is normal in $G$ if and only if $Kg = gK$ for every $g\in G$,
	for associativity of the multiplication of nonempty subsets gives $K = (Kg)g^{-1} = gKg^{-1}$.
\end{remark}

\begin{definition}
	If $x\in G$, then a \term{conjugate} of $x$ in $G$ is an element of the form $axa^{-1}$ for some $a\in G$;
	equivalently, $x$ and $y$ are conjugate if $y=\gamma{_a}(x)$ for some $a\in G$.
\end{definition}


\section{Quotient Groups}

\begin{theorem}
	If $N\vartriangleleft G$, then the cosets of $N$ in $G$ form a group, denoted by $\faktor{G}{N}$, of order $[G:N]$.
\end{theorem}

\begin{corollary}
	If $N\vartriangleleft G$, then the \term{natural map} (i.e., the function $\nu:G\to\faktor{G}{N}$ defined by $\nu(a)=Na$) is a surjective homomorphism with kernel $N$.
\end{corollary}

\begin{definition}
	If $a,b\in G$, the \term{commutator} of $a$ and $b$, denoted by $[a,b]$, is \[ [a,b]=aba^{-1}b^{-1}. \]
	The \term {commutator subgroup} (or \textit{derived subgroup}) of $G$, denoted by $G'$, is the subgroup of $G$ generated by all the commutators.
\end{definition}

\begin{theorem}
	The commutator subgroup $G'$ is a normal subgroup of $G$. Moreover, if $H\vartriangleleft G$, then $\faktor{G}{H}$ is abelian if and only if $G'\leq H$.
\end{theorem}


\section{The Isomorphism Theorems}

\begin{theorem}[First Isomorphism Theorem]
	Let $f:G\to H$ be a homomorphism with kernel $K$. Then $K$ is a normal subgroup of $G$ and $\faktor{G}{K}\cong \mathrm{im}\;f$.
\end{theorem}

\begin{lemma}
	If $S$ and $T$ are subgroups of $G$ and if one of them is normal, then $ST=S\vee T=TS$.
\end{lemma}

\begin{theorem}[Second Isomorphism Theorem]
	Let $N$ and $T$ be subgroups of $G$ with $N$ normal. Then $N\cap T$ is normal in $T$ and $\faktor{T}{N\cap T}\cong\faktor{NT}{N}$.
\end{theorem}

\begin{theorem}[Third Isomorphism Theorem]
	Let $K\leq H\leq G$, where both $K$ and $H$ are subgroups of $G$. Then $\faktor{H}{K}$ is a normal subgroup of $\faktor{G}{K}$ and
	\[ \faktor{\left(\faktor{G}{K}\right)}{\left(\faktor{H}{K}\right)}\cong\faktor{G}{H}. \]
\end{theorem}



\section{The Correspondence Theorem}

\begin{theorem}[Correspondence Theorem]
	Let $K\vartriangleleft G$ and let $\nu:G\mapsto\faktor{G}{K}$ be the natural map.
	Then $S\mapsto\nu(S)=\faktor{S}{K}$ is a bijection from the family of all those subgroups $S$ of $G$ which contain $K$
	to the family of all the subgroups of $\faktor{G}{K}$.
	\par
	Moreover, if we denote $\faktor{S}{K}$ by $S^\ast$, then:
	\begin{enumerate}
		\item $T\leq S$ if and only if $T^\ast\leq S^\ast$, and then $[S:T]=[S^\ast:T^\ast]$; and
		\item $T\vartriangleleft S$ if and only if $T^\ast\vartriangleleft S^\ast$, and then $\faktor{S}{T}\cong\faktor{S^\ast}{T^\ast}$.
	\end{enumerate}
\end{theorem}

\begin{definition}
	A subgroup $H\leq G$ is a \term{maximal normal subgroup} of $G$ if there is no normal subgroup $N$ of $G$ with $H<N<G$.	
\end{definition}

\begin{definition}
	A group $G\neq 1$ is \term{simple} if it has no normal subgroups other than $G$ and $1$.
\end{definition}



\section{Direct Products}

\begin{definition}
	If $H$ and $K$ are groups, then their \term{direct product}, denoted by $H\times K$, is the group with elements all
	ordered pairs $(h, k)$, where $h\in H$ and $k\in K$, and with operation $(h, k)(h', k')=(hh', kk')$.
\end{definition}

\begin{theorem}
	Let $G$ be a group with normal subgroups $H$ and $K$. If $HK=G$ and $H\cap K=1$, then $G\cong H\times K$.
\end{theorem}

\begin{theorem}
	If $A\vartriangleleft H$ and $B\vartriangleleft K$ then $A\times B\vartriangleleft H\times K$ and
	$\faktor{H\times K}{A\times B}\cong\faktor{H}{A}\times\faktor{K}{B}$.
\end{theorem}

\begin{corollary}
	If $G=H\times K$, then $\faktor{G}{H\times 1}\cong K$.
\end{corollary}


\chapter{Symmetric Groups and \textit{G}-Sets}

\section{Conjugates}
\begin{lemma}
	If $G$ is a group, then the relation “y is a \term{conjugate} of $x$ in $G$,” that is, $y=gxg^{-1}$ for some $g\in G$, is an equivalence relation.
\end{lemma}

\begin{definition}
	If $G$ is a group, then the equivalence class of $a\in G$ under the relation “y is a conjugate of $x$ in $G$” is called the \term{conjugacy class} of $a$; it is denoted by $a^G$.
\end{definition}

\begin{definition}
	The \term{center} of a group $G$, denoted by $Z(G)$, is the set of all $a\in G$ at commute with every element of $G$.
\end{definition}

\begin{definition}
	If $a\in G$, then the \term{centralizer} of $a$ in $G$, denoted by $C_G(a)$, is the set of all $x\in G$ which commute with $a$.
\end{definition}

\begin{theorem}
	If $a\in G$, the number of conjugates of $a$ is equal to the index of its centralizer:\
	$|a^G|=[G:C_G(a)]$, and this number is a divisor of $|G|$ when $G$ is finite.
\end{theorem}

\begin{definition}
	If $H\leq G$ and $g\in G$, then the \term{conjugate} $gHg^{-1}$ is $\{ghg^{-1}:h\in H\}$. The conjugate $gHg^{-1}$ is often denoted by $H^g$.
\end{definition}

\begin{definition}
	If $H\leq G$, then the \term{normalizer} of $H$ in $G$, denoted by $N_G(H)$, is
	\[N_G(H)=\{a\in G:aHa^{-1}=H\}.\]
\end{definition}

\begin{theorem}
	If $H\leq G$, then the number $c$ of conjugates of $H$ in $G$ is equal to the index of its normalizer: $c=[G:N_G(H)]$, and $c$ divides $|G|$ when $G$ is finite.
	Moreover, $aHa^{-1}=bHb^{-1}$ if and only if $b^{-1}a\in N_G(H)$.
\end{theorem}

\section{Symmetric Groups}

\begin{definition}
	Two permutations $\alpha,\beta\in S_n$ have the \term{same cycle structure} if their complete factorizations into disjoint
	cycles have the same number of $r$-cycles for each $r$.
\end{definition}

\begin{lemma}
	If $\alpha,\beta\in S_n$, then $\alpha\beta\alpha^{-1}$ is the permutation with the same cycle structure as $\beta$ which is
	obtained by applying $\alpha$ to the symbols in $\beta$.
\end{lemma}

\begin{theorem}
	Permutations $\alpha,\beta\in S_n$ are conjugate if and only if they have the same cycle structure.
\end{theorem}

\begin{corollary}
	A subgroup $H$ of $S_n$ is a normal subgroup if and only if, whenever $\alpha\in H$, then every $\beta$ having the same
	cycle structure as $\alpha$ also lies in $H$.
\end{corollary}

\begin{theorem}
	$A_4$ is a group of order 12 having no subgroup of order 6.
\end{theorem}

\begin{definition}
	If $n$ is a positive integer, then a \term{partition of n} is a sequence of integers $1\leq i_1\leq i_2\leq\cdots\leq i_r$ with $\sum i_j = n$.
\end{definition}

\section{The Simplicity of $A_n$}

\begin{lemma}
	$A_5$ is simple.
\end{lemma}

\begin{lemma}
	Let $H\vartriangleleft A_n$, where $n\geq 5$. If $H$ contains a 3-cycle, then $H = A_n$.
\end{lemma}

\begin{lemma}
	$A_6$ is simple.
\end{lemma}

\begin{theorem}
	$A_n$ is simple for all $n\geq 5$.
\end{theorem}

\section{Some Representation Theorems}
\begin{theorem}[Cayley, 1878]
	Every group $G$ can be imbedded as a subgroup of $S_G$. In particular, if $|G|=n$, then $G$ can be imbedded in $S_n$.
\end{theorem}

\begin{definition}
	The homomorphism $L:G\to S_G$, given by $a\mapsto L_a$, is called the \term{left regular representation} of $G$.
\end{definition}

\begin{corollary}
	If $k$ is a field and $G$ is a finite group of order $n$, then $G$ can be imbedded in $\mathrm{GL}(n,k)$.
\end{corollary}

\begin{theorem}
	If $H\leq G$ and $[G:H]=n$, then there is a homomorphism $\rho:G\to S_n$ with $\mathrm{ker}\,\rho\leq H$.
\end{theorem}

\begin{definition}
	The homomorphism $\rho$ in Theorem 3.14 is called the \term{representation of G on the cosets of H}.
\end{definition}

\begin{corollary}
	An simple group $G$ which contains a subgroup $H$ of index $n$ can be imbedded in $S_n$.
\end{corollary}

\begin{corollary}
	An infinite simple group $G$ has no proper subgroups of finite index.
\end{corollary}

\begin{theorem}
	Let $H\leq G$ and let $X$ be the family of all the conjugates of $H$ in $G$. There is a homomorphism $\psi:G\to S_X$ with $\mathrm{ker}\,\psi\leq N_G(H)$.
\end{theorem}

\begin{definition}
	The homomorphism $\psi$ of Theorem 3.17 is called the \term{representation of G on the conjugates of H}.
\end{definition}

\section{\textit{G}-Sets}
\begin{definition}
	If $X$ is a set and $G$ is a group, then $X$ is a \term{G-set} if there is a function $\alpha:G\times X\to X$ (called an \term{action}), denoted by $\alpha:(g,x)\mapsto gx$, such that:
	\begin{enumerate}
		\item $1x=x$ for all $x\in X$; and
		\item $g(hx) = (gh)x$ for all $g,h\in G$ and $x\in X$.
	\end{enumerate}
	One also says that $G$ \term{acts} on $X$. If $|X|=n$, then $n$ is called the \term{degree} of the \textit{G}-set $X$.
\end{definition}

\begin{theorem}
	If $X$ is a G-set with action $\alpha$, then there is a homomorphism $\tilde{\alpha}:G\to S_X$ given by $\tilde{\alpha}(g):x\mapsto gx=\alpha(g,x)$. Conversely, every homomorphism $\varphi:G\to S_X$ defines an action, namely, $gx=\varphi(g)x$, which makes $X$ into a G-set.
\end{theorem}

\begin{definition}
	If $X$ is a \textit{G}-set and $x\in X$, then the \term{G-orbit} of $x$ is $\mathcal{O}(x)=\{gx:g\in G\}\subset X$.
\end{definition}

\begin{definition}
	If $X$ is a \textit{G}-set and $x\in X$, then the \term{stabilizer} of $x$, denoted by $G_x$, is the subgroup $G_x=\{g\in G:gx=x\}\leq G$.
\end{definition}

\begin{theorem}
	If $X$ is a \textit{G}-set and $x\in X$, then $|\mathcal{O}(x)|=[G:G_x]$.
\end{theorem}

\begin{corollary}
	If a finite group $G$ acts on a set $X$, then the number of elements in any orbit is a divisor of $|G|$.
\end{corollary}

\begin{corollary}
	\quad
	\begin{enumerate}
		\item If $G$ is a finite group and $x\in G$, then the number of conjugates of $x$ in $G$ is $[G:C_G(x)]$.
		\item If $G$ is a finite group and $H\leq G$, then the number of conjugates of $H$ in $G$ is $[G:N_G(H)]$.
	\end{enumerate}
\end{corollary}

\begin{definition}
	A \textit{G}-set $X$ is \term{transitive} if it has only one orbit; that is, for every $x,y\in X$, there exists $\sigma\in G$ with $y=\sigma x$.
\end{definition}

\section{Some Geometry}
\begin{definition}
	The \term{dihedral group} $D_{2n}$, for $2n\geq 4$, is a group of order $2n$ which is genrerated by two elements $s$ and $t$ such that $s_n=1$, $t^2=1$, and $tst=s^{-1}$.\\
	\\
	Note that $D_{2n}$ is not abelian for all $n\geq 3$, while $D_4$ is the Klein 4-group $\mathbf{V}$.
\end{definition}

\chapter{The Sylow Theorems}

\section{\textit{p}-Groups}
\begin{definition}
	If $p$ is a prime, then a \term{p-group} is a group in which every element has order a power of $p$.
\end{definition}

\begin{lemma}
	If $G$ is a finite abelian group whose order is divisible by a prime $p$, then $G$ contains an element of order $p$.
\end{lemma}

\begin{theorem}[Cauchy, 1845]
	If $G$ is a finite group whose order is divisible by a prime $p$, then $G$ contains an element of order $p$.
\end{theorem}

\begin{definition}
	The following is called the \term{class equation} of the finite group $G$.
	\[ |G| = |Z(G)| + \sum_i \; \left[ G : C_G(x_i) \right] \]
	$Z(G)$ consists of all the elements of $G$ whose conjugacy class has just one element; each $x_i$ is selected from each conjugacy class with more than one element.
\end{definition}

\begin{corollary}
	A finite group $G$ is a p-group if and only if $|G|$ is a power of $p$.
\end{corollary}

\begin{theorem}
	If $G\neq 1$ is a finite p-group, then its center $Z(G)\neq 1$.
\end{theorem}

\begin{corollary}
	If $p$ is a prime, then every group $G$ of order $p^2$ is abelian.
\end{corollary}

\begin{theorem}
	Let $G$ be a finite p-group.
	\begin{enumerate}
		\item If $H$ is a proper subgroup of $G$, then $H<N_G(H)$.
		\item Every maximal subgroup of G is normal and has index $p$.
	\end{enumerate}
\end{theorem}

\begin{lemma}
	If $G$ is a finite p-group and $r_1$ is the number of subgroups of $G$ having order $p$, then $r_1\equiv 1\mod p$.
\end{lemma}

\begin{theorem}
	If $G$ is a finite p-group and $r_s$ is the number of subgroups of $G$ having order $p^s$, then $r_s\equiv 1\mod p$.
\end{theorem}

\begin{lemma}[Landau, 1903]
	Given $n>0$ and $q\in\mathbb{Q}$, there are only finitely many $n$-tuples $(i_1,\,\ldots\,,i_n)$ of positive integers such that $q=\sum_{j=1}^n (1/i_j)$.
\end{lemma}

\begin{theorem}
	For every $n\geq 1$, there are only finitely many groups having exactly $n$ conjugacy classes.
\end{theorem}

\section{The Sylow Theorems}
\begin{definition}
	If $p$ is a prime, then a \term{Sylow p-subgroup} $P$ of a group $G$ is a maximal $p$-subgroup.
\end{definition}

\begin{lemma}
	Let $P$ be a Sylow $p$-subgroup of a finite group $G$.
	\begin{enumerate}
		\item $\left|\faktor{N_G(P)}{P}\right|$ is prime to $p$.
		\item If $a\in G$ has order some power of $p$ and $aPa^{-1}=P$, then $a\in P$.
	\end{enumerate}
\end{lemma}

\begin{theorem}[Sylow, 1872]
	\quad
	\begin{enumerate}
		\item If $P$ is a Sylow $p$-subgroup of a finite group $G$, then all Sylow $p$-subgroups of $G$ are conjugate to $P$.
		\item If there are $r$ Sylow $p$-subgroups, then $r$ is a divisor of $|G|$ and $r\equiv 1\mod p$.
	\end{enumerate}
\end{theorem}

\begin{corollary}
	A finite group $G$ has a unique Sylow $p$-subgroup $P$, for some prime $p$, if and only if $P\vartriangleleft G$.
\end{corollary}

\begin{theorem}
	If $G$ is a finite group of order $p{^e}m$, where $\gcd(p,m)=1$, then every Sylow $p$-subgroup $P$ of $G$ has order $p^e$.
\end{theorem}

\begin{corollary}
	Let $G$ be a finite group and let $p$ be prime. If $p^k$ divides $|G|$, then $G$ contains a subgroup of order $p^k$.
\end{corollary}

\begin{lemma}
	If $p$ is a prime not dividing an integer $m$, then for all $n\geq 1$, the binomial coefficient $\binom{p{^n}m}{p^n}$ is not divisible by $p$.
\end{lemma}

\begin{theorem}[Wielandt's Proof]
	If $G$ is a finite group of order $p{^n}m$, where $\gcd(p,m)=1$, then $G$ has a subgroup of order $p^n$.
\end{theorem}

\begin{theorem}[Frattini Argument]
	Let $K$ be a normal subgroup of a finite group $G$. If $P$ is a Sylow $p$-subgroup of $K$ (for some prime $p$), then $G=KN_G(P)$.
\end{theorem}

\begin{definition}
	An $n\times n$ matrix $A$ over a commutative ring $R$ is \term{unitriangular} if it has $0$s below the diagonal and $1$s on the diagonal.
	The set of all unitriangular $3\times 3$ matrices over $\mathbb{Z}_p$ is denoted by $\mathrm{UT}(3,\mathbb{Z}_p)$.
\end{definition}
\end{document}
