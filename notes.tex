\documentclass[12pt]{report}
\usepackage[top=0.75in, bottom=1in, left=1in, right=1in]{geometry}
\usepackage{amsmath, amssymb, amsthm}
\title{Group Theory Notes}

\newtheorem{theorem}{Theorem}[chapter]
\newtheorem{corollary}[theorem]{Corollary}
\newtheorem{lemma}[theorem]{Lemma}
\theoremstyle{definition}
\newtheorem*{definition}{Definition}
\newtheorem*{notation}{Notation}
\newtheorem*{remark}{Remark}

\newcommand{\term}[1]{\textbf{\textit{#1}}}
\newcommand{\gen}[1]{{\langle}#1{\rangle}}

\begin{document}
\chapter{Groups and Homomorphisms}
\section{Permutations}
\begin{definition}
	If $X$ is a nonempty set, a \term{permutation} of $X$ is a bijection $\alpha:X\to X$. We denote the set of all permutations of $X$ by $S_X$.
\end{definition}

\section{Cycles}
\begin{definition}
	If $x\in X$ and $\alpha\in S_X$, then $\alpha$ \term{fixes} $x$ if $\alpha(x)=x$ and $\alpha$ \term{moves} $x$ if $\alpha(x)\neq x$.
\end{definition}

\begin{definition}
	Let $i_1, i_2, \dots, i_r$ be distinct integers between $1$ and $n$. If $\alpha\in S_n$ fixes the remaining $n-r$ integers and if
	\[ \alpha(i_1)=i_2,\;\alpha(i_2)=i_3,\;\dots,\;\alpha(i_{r-1})=i_r,\;\alpha(i_r)=i_1, \]
	then $\alpha$ is an \term{r-cycle}; one also says that $\alpha$ is a cycle of \term{length} r. Denote $\alpha$ by $(i_1\enspace i_2\enspace \cdots \enspace i_r)$.
	\smallskip

	Every 1-cycle fixes every element of $X$, and so all 1-cycles are equal to the identity. A 2-cycle, which merely interchanges a pair of elements, is called a \term{transposition}.
\end{definition}

\begin{definition}
	Two permutations $\alpha, \beta\in S_X$ are \term{disjoint} if every $x$ moved by one is fixed by the other.
	In symbols, if $\alpha(x)\neq x$, then $\beta(x)=x$ and if $\beta(y)\neq y$, then $\alpha(y)=y$
	(of course it is possible that there is $z\in X$ with $\alpha(z)=z=\beta(z)$).
	A family of permutations $\alpha_1,\alpha_2,\dots,\alpha_m$ is \term{disjoint} if each pair of them is disjoint.
\end{definition}

\section{Factorization into Disjoint Cycles}
\begin{theorem}
	Every permutation $alpha\in S_n$ is either a cycle or a product of disjoint cycles.
\end{theorem}

\begin{definition}
	A \term{complete factorization} of a permutation $\alpha$ is a factorization of $\alpha$ as a product of disjoint cycles which contains one 1-cycle $(i)$ for every $i$ fixed by $\alpha$.
	\smallskip

	In a complete factorization of a permutation $\alpha$, every $i$ between $1$ and $n$ occurs in exactly one of the cycles.
\end{definition}

\begin{theorem}
	Let $\alpha\in S_n$ and let $\alpha=\beta_1\dots\beta_t$ be a complete factorization into disjoint cycles. This factorization is unique except for the order in which the factors occur.
\end{theorem}

\section{Even and Odd Permutations}
\begin{theorem}
	Every permutation $\alpha\in S_n$ is a product of transpositions. For each cycle of length $r>1$:
	\[ (i_1\enspace i_2\enspace\dots\enspace i_r)=(i_1\enspace i_r)(i_1\enspace i_{r-1})\dots(i_1\enspace i_2)\]
\end{theorem}

\begin{definition}
	A permutation $\alpha\in S$ is \term{even} if it is a product of an even number of transpositions; otherwise, $alpha$ is \term{odd}.
\end{definition}

\begin{lemma}
	If $k, l\geq0$, then
	\[ (a\enspace b)(a\enspace c_1\enspace\dots\enspace c_k\enspace b\enspace d_1\enspace\dots\enspace d_l) = (a\enspace c_1\enspace\dots\enspace c_k)(b\enspace d_1\enspace\dots\enspace d_l) \]
	and
	\[ (a\enspace b)(a\enspace c_1\enspace\dots\enspace c_k)(b\enspace d_1\enspace\dots\enspace d_l) = (a\enspace c_1\enspace\dots\enspace c_k\enspace b\enspace d_1\enspace\dots\enspace d_l) \]
\end{lemma}

\begin{definition}
	If $\alpha\in S_n$ and $\alpha=\beta_1\dots\beta_t$ is a complete factorization into disjoint cycles, then \term{signum} $\alpha$ is defined by
	\[ \text{sgn}(\alpha) = (-1)^{n-t}. \]
\end{definition}

\begin{lemma}
	If $\beta\in S_n$ and $\tau$ is a transposition, then
	\[ \textnormal{sgn}(\tau\beta) = -\textnormal{sgn}(\beta). \]
\end{lemma}

\begin{lemma}
	For all $\alpha, \beta\in S_n$,
	\[ \textnormal{sgn}(\alpha\beta) = \textnormal{sgn}(\alpha)\,\textnormal{sgn}(\beta). \]
\end{lemma}

\begin{theorem}
	\quad
	\renewcommand{\labelenumi}{\textnormal{(\roman{enumi})}}
	\begin{enumerate}
		\item A permutation $\alpha\in S_n$ is even if and only if $\textnormal{sgn}(\alpha)=1$.
		\item A permutation $\alpha\in S_n$ is odd if and only if it is a product of an odd number of transpositions.
	\end{enumerate}
\end{theorem}

\section{Semigroups}
\begin{definition}
	A (binary) \term{operation} on a nonempty set $G$ is a function $\mu:G\times G\to G$.
\end{definition}

\begin{definition}
	An operation $\ast$ on a set $G$ is \term{associative} if \[(a\ast b)\ast c=a\ast(b\ast c) \] for every $a, b, c\in G$.
\end{definition}

\begin{definition}
	An expression $a_1\ast a_2\ast\cdots\ast a_n$ \term{needs no parentheses} if, no matter what choices of multiplications of adjacent factors are made, the resulting elements of $G$ are all equal.
\end{definition}

\begin{theorem}[Generalized Associativity]
	If $\ast$ is an associative operation on a set $G$, then every expression $a_1\ast a_2\ast\cdots\ast a_n$ needs no parentheses.
\end{theorem}

\begin{definition}
	A \term{semigroup} $(G,\ast)$ is a nonempty set $G$ equipped with an associative operation $\ast$.
\end{definition}

\begin{definition}
	Let $G$ be a semigroup and let $a\in G$. Define $a^1=a$ and, for $n\geq1$, define $a^{n+1}=a\ast a^n$.
\end{definition}

\begin{corollary}
	Let $G$ be a semigroup, let $a\in G$, and let $m$ and $n$ be positive integers. Then $a^m\ast a^n=a^{m+n}=a^n\ast a^m$ and $(a^m)^n=a^{mn}=(a^n)^m$.
\end{corollary}


\section{Groups}
\begin{remark}
	The four permutations $\{1,\;(1\enspace2)(3\enspace4),\;(1\enspace3)(2\enspace4),\;(1\enspace4)(2\enspace3) \}$
	form a group $\mathbf{V}$ called the \term{Klein 4-group}.
\end{remark}

\section{Homomorphisms}

\chapter{The Isomorphism Theorems}
\section{Subgroups}
\begin{definition}
	A nonempty subset $S$ of a group $G$ is a \term{subgroup} of $G$
	if $s\in S$ implies $s^{-1}\in S$ and $s, t\in S$ imply $st\in S$.
\end{definition}

\begin{theorem}
	If $S\leq G$ (i.e., if $S$ is a subgroup of G), then $S$ is a group in its own right.
\end{theorem}

\begin{theorem}
	A subset $S$ of a group $G$ is a subgroup if and only if $1\in S$ and $s, t\in S$ imply $st^{-1}\in S$.
\end{theorem}

\begin{definition}
	If $G$ is a group and $a\in G$, then the \term{cyclic subgroup generated by a}, denoted by $\gen{a}$, is the set of all the powers of $a$.
	A group $G$ is called \term{cyclic} if there is $a\in G$ with $G=\gen{a}$; that is, $G$ consists of all the powers of $a$.
\end{definition}

\begin{definition}
	If $G$ is a group and $a\in G$, then the \term{order} of $a$ is $|\gen{a}|$, the number of elements in $\gen{a}$.
\end{definition}

\begin{theorem}
	If $G$ is a group and $a\in G$ has finite order $m$, then $m$ is the smallest positive integer such that $a^m=1$.
\end{theorem}

\begin{corollary}
	If $G$ is a finite group, then a nonempty subset $S$ of $G$ is a subgroup if and only if $s, t\in S$ imply $st \in S$.
\end{corollary}

\begin{definition}
	Let $f:G\to H$ be a homomorphism, then $\term{kernel}\ f = \{ a\in G: f(a)=1\}$ and $\term{image}\ f = \{ h\in H: h=f(a) \text{ for some } a\in G \}$.
\end{definition}

\begin{notation}
	We usually write \term{ker f} instead of kernel $f$ and \term{im f} instead of image $f$.
\end{notation}

\begin{theorem}
	The intersection of any family of subgroups of a group $G$ is again a subgroup of $G$.
\end{theorem}

\begin{corollary}
	If $X$ is a subset of a group $G$, then there is a \textbf{smallest} subgroup $H$ of $G$ containing $X$; that is, if $X\subset S$ and $S\leq G$, then $H\leq S$.
\end{corollary}

\begin{definition}
	If $X$ is a subset of a group $G$, then the smallest subgroup of $G$ containing $X$, denoted by $\gen{X}$, is called the \term{subgroup generated by X}.
	One also says that $X$ \term{generates} $\gen{X}$.
\end{definition}

\begin{notation}
	If $H$ and $K$ are subgroups of $G$, then the subgroup $\gen{H\cup K}$ is denoted by $H\vee K$.
\end{notation}

\begin{definition}
	If $X$ is a nonempty subset of a group $G$, then a \term{word} on X is an element $w\in G$ of the form
	\[ w = x_1^{e_1} x_2^{e_2} \cdots x_n^{e_n}, \]
	where $x_i\in X$, $e_i=\pm1$, and $n\geq1$.
\end{definition}

\begin{theorem}
	Let $X$ be a subset of a group $G$. If $X=\varnothing$, then $\gen{X}=1$; if $X$ is nonempty, then $\gen{X}$ is the set of all the words on $X$.
\end{theorem}

\section{Lagrange's Theorem}
\begin{definition}
	If $S$ is a subgroup of $G$ and if $t\in G$, then a \term{right coset} of $S$ in $G$ is the subset of $G$
	\[ St = \{ st: s\in S \} \]
	(a \term{left coset} is $tS = \{ ts: s\in S \}$). One calls $t$ a \term{representative} of $St$ (and also of $tS$).
\end{definition}

\begin{lemma}
	If $S\leq G$, then $Sa=Sb$ if and only if $ab^{-1}\in S$ ($aS=bS$ if and only if $b^{-1}a\in S$).
\end{lemma}

\begin{theorem}
	If $S\leq G$, then any two right (or any two left) cosets of $S$ in $G$ are either identical or disjoint.
\end{theorem}

\begin{theorem}
	If $S\leq G$, then the number of right cosets of $S$ in $G$ is equal to the number of left cosets of $S$ in $G$.
\end{theorem}

\begin{definition}
	If $S\leq G$, then the \term{index} of $S$ in $G$, denoted by $[G:S]$, is the number of right cosets of $S$ in $G$.
\end{definition}

\begin{definition}
	If $G$ is a group, then the \term{order} of $G$, denoted by $|G|$, is the number of elements in $G$.
\end{definition}

\begin{theorem}[Lagrange]
	If $G$ is a finite group and $S\leq G$, then $|S|$ divides $|G|$ and $[G:S] = |G|/|S|$.
\end{theorem}

\begin{corollary}
	If $G$ is a finite group and $a\in G$, then the order of $a$ divides $|G|$.
\end{corollary}

\begin{definition}
	A group $G$ has \term{exponent} n if $x^n=1$ for all $x\in G$.
\end{definition}

\begin{corollary}
	If $p$ is a prime and $|G|=p$, then $G$ is a cyclic group.
\end{corollary}

\begin{corollary}[Fermat]
	If $p$ is a prime and $a$ is an integer, then $a^p\equiv a\mod p$.
\end{corollary}

\begin{notation}
	The multiplicative group of elements of $\mathbb{Z}_n$ is denoted by $G = U(\mathbb{Z}_n)$. $G$ is a group of order $n-1$.
\end{notation}

\section{Cyclic Groups}

\begin{definition}
	The \term{Euler \varphi-function}, also called \term{Euler's totient function} is defined as follows:
	\[ \varphi(1)=1;\quad\text{ if }n>1,\text{ then }\varphi(n)=|\{k:1\leq k<n\text{ and }\gcd(k, n)=1\}|. \]
	In other words, $\varphi(n)$ is the number of positive integers that are relatively prime to $n$.
\end{definition}

\begin{lemma}
	If $G$ is a cyclic group of order $n$, then there exists a unique subgroup of order $d$ for every divisor $d$ of $n$.
\end{lemma}

\begin{theorem}
	If $n$ is a positive integer, then
	\[ n=\sum_{d\mid n} \varphi(d), \]
	where the sum is over all divisors $d$ of $n$ with $1\leq d\leq n$.
\end{theorem}

\begin{theorem}
	A group $G$ of order $n$ is cyclic if and only if, for each divisor $d$ of $n$, there is at most one cyclic subgroup of $G$ having order $d$.
\end{theorem}

\begin{theorem}
	\quad
	\renewcommand{\labelenumi}{\textnormal{(\roman{enumi})}}
	\begin{enumerate}
		\item If $F$ is a field and if $G$ is a finite subgroup of $F^\times$, the multiplicative group of nonzero elements of $F$, then $G$ is cyclic.
		\item If $F$ is a finite field, then its multiplicative group $F^\times$ is cyclic.
	\end{enumerate}
\end{theorem}

\begin{theorem}
	Let $p$ be a prime. A group $G$ of order $p^n$ is cyclic if and only if it is an abelian group having a unique subgroup of order $p$.
\end{theorem}
\end{document}
