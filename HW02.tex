\documentclass[12pt]{article}
\usepackage[top=1.0in, bottom=0.75in, left=0.75in, right=2.5in]{geometry}
\usepackage{fancyhdr, textpos}
\usepackage{amsmath, amssymb, amsthm}
\usepackage{ebgaramond-maths, plex-mono}
\usepackage[makeroom]{cancel}

\pagestyle{fancy}
\fancyhf{}
\setlength\headheight{28pt}
\setlength\headwidth{7in}
\setlength{\TPHorizModule}{1pt}
\setlength{\TPVertModule}{1pt}
\lhead{\textsc{group theory\\spring 2018}}
\rhead{\textsc{homework bundle \#2\\page \thepage}}
\chead{\textsc{rob\\ireton}}

\newcommand{\zee}{\mathbb{Z}}

\newenvironment{exercise}[2]{\begin{textblock}{32}[1,0](0,#2)\noindent#1\end{textblock}}{\vspace{1in}}

\begin{document}
% \begin{exercise}{1.16}{6.2}
% 	{\noindent}Let $\alpha\in S_n$ and $p\in \mathbb{N}$ be prime. Prove that if $\alpha^p=1$,
% 	then either $\alpha=1$, $\alpha$ is a $p$-cycle, or $\alpha$ is a product of disjoint $p$-cycles.
% 	\bigskip
%
% \end{exercise}

\begin{exercise}{1.18}{6.2} %OK
	{\noindent}Give an example of $\alpha, \beta, \gamma\in S_5$ such that:
	\begin{itemize}
		\item $\alpha$ commutes with $\beta$
		\item $\beta$ commutes with $\gamma$
		\item $\alpha$ and $\gamma$ do not commute.
	\end{itemize}
	\bigskip
	Based on the results of Exercise 1.8---that if $\alpha$ and $\beta$ are disjoint permutations,
	then they commute---choose $\alpha, \beta, \gamma\in S_5$ such that $\alpha, \beta$ disjoint,
	$\beta, \gamma$ disjoint, $\alpha, \gamma$ not disjoint. e.g.:
	$\alpha = (1\enspace2)$;\quad $\beta = (3\enspace4)$;\quad $\gamma = (1\enspace5)$.
\end{exercise}


% \begin{exercise}{1.21}{6.2}
%   {\noindent}Show that $S_n$ has the same number of even permutations as odd permutations.
% 	[Hint: If $\tau$ is a transposition, consider the function $f : S_n\to S_n$ given by $f(\alpha)=\tau\alpha$.]
% 	\bigskip
%
% \end{exercise}


\begin{exercise}{1.26}{3.5} %OK
	{\noindent}Prove: a group $G$ in which $\forall x\in G$, $x^2=e$ must be abelian.
	\bigskip

	\begin{proof}
		Let $a, b\in G$. Observe that $b^2 = bb = e$. Multiplying by $a$ on the left and on the right gives
		$abba = aea = e$. Next observe that $ab\in G$, so $(ab)(ab) = e$, then we have $abab = e = abba$. The result of
		exercise 1.6 lets us cancel, giving $\cancel{a}\cancel{b}ab = \cancel{a}\cancel{b}ba$. Since $ab = ba\enspace
		\forall a, b \in G$, G is, by definition, abelian.
	\end{proof}
\end{exercise}


\begin{exercise}{1.39}{5}
	{\noindent}Let $f : X\to Y$ be a bijection between sets $X$ and $Y$. Show that the mapping given by \[\alpha \mapsto f\circ\alpha\circ f^{-1}\]
	is an isomorphism $S_X\to S_Y$.
	\bigskip

	{\noindent}First, some clarification: if $f:X\to Y$, then $f^{-1}:Y\to X$ is the inverse of $f$ and $ff^{-1}=(1)=f^{-1}f$.
	Based on context, assume $\alpha:X\to X$ is a permutation in $S_X$. That means that $f\circ\alpha\circ f^{-1}:Y\to Y$ is a
	permutation in $S_Y$. Let's call our mapping $\varphi:S_X\to S_Y$.\\
	\\
	Let $a,b\in S_X$. Then
	\begin{align*}
		\varphi(a\circ b) &= f\circ a\circ b\circ f^{-1} \\
		                  &= f\circ a\circ(1)\circ b\circ f^{-1} \\
											&= f\circ a\circ f^{-1}\circ f\circ b\circ f^{-1} \\
											&= (f\circ a\circ f^{-1})\circ(f\circ b\circ f^{-1}) \\
											&= \varphi(a)\circ\varphi(b)
	\end{align*}
	So, $\varphi$ is a homomorphism.\\
	\\
	Next, since $\alpha$ is a permutation, it is a bijection. It follows that $\varphi$ is also a bijection, because a composition of bijections is
	itself a bijection. ({\small{\textit{c.f.} https://goo.gl/QuJkb3}})\\
	\\
\end{exercise}


\begin{exercise}{1.29}{9} %OK
	{\noindent}Consider $\alpha: \zee_{11}\to \zee_{11}$ given by $\alpha(x)=4x^2-3x^7$.
	\begin{itemize}
		\item Show that $\alpha$ is a permutation of $\zee_{11}$ and write it as a product of disjoint cycles.
		\bigskip

		A Julia expression:
		\begin{verbatim}
			julia> string([n => 11+(4n^2-3n^7)%11 for n=1:11])
			"Pair{Int64,Int64}[1=>12, 2=>6, 3=>9, 4=>5, 5=>3,
			6=>10, 7=>2, 8=>8, 9=>4, 10=>7, 11=>11]"
		\end{verbatim}
		with results adjusted to use $1$--$11$ as class representitives, shows that, in two-rowed notation,
		\[
		\alpha = \left(
			\begin{array}{rrrrrrrrrrr}
		    1 &  2 &  3 &  4 &  5 &  6 &  7 &  8 &  9 & 10 & 11 \\
		    1 &  6 &  9 &  5 &  3 & 10 &  2 &  8 &  4 &  7 & 11
		  \end{array}
			\right)
		\]
		We can verify by inspection that $\alpha$ is a permutation. As a product of disjoint cycles, we have:
		\[ \alpha = (1)(2\enspace6\enspace10\enspace7)(3\enspace9\enspace4\enspace5)(8)(11) = (2\enspace6\enspace10\enspace7)(3\enspace9\enspace4\enspace5) \]

		\item What is the parity of $\alpha$?
		\bigskip

		Using Theorem 1.3, We can express $\alpha$ as a product of transpositions:
		\[ \alpha = (2\enspace7)(2\enspace10)(2\enspace6)(3\enspace5)(3\enspace4)(3\enspace9) \]
		There are an even number of transpositions, so, by definition, $\alpha$ is even.

		\item What is $\alpha^{-1}$?
		\bigskip

		$\alpha^{-1}=(2\enspace7\enspace10\enspace6)(3\enspace5\enspace4\enspace9)$.
	\end{itemize}
\end{exercise}

\newpage

\begin{exercise}{1.36}{3.5}
	{\noindent}Prove that the following four permutations form a group, denoted $\mathbf{V}$, called the \\(Klein) Four-group.
	\[(1);\quad(1\enspace2)(3\enspace4);\quad(1\enspace3)(2\enspace4);\quad(1\enspace4)(2\enspace3).\]
	\bigskip

	\begin{proof}
		The `multiplication' table for $\mathbf{V}$ is given by:
		\bigskip

		\setlength{\tabcolsep}{1em} % for the horizontal padding
		\renewcommand{\arraystretch}{1.25}% for the vertical padding
		\begin{tabular}{r|rrrr}
			$\circ$                    & $(1)$                      & $(1\enspace2)(3\enspace4)$ & $(1\enspace3)(2\enspace4)$ & $(1\enspace4)(2\enspace3)$ \\ \hline
			$(1)$                      & $(1)$                      & $(1\enspace2)(3\enspace4)$ & $(1\enspace3)(2\enspace4)$ & $(1\enspace4)(2\enspace3)$ \\
			$(1\enspace2)(3\enspace4)$ & $(1\enspace2)(3\enspace4)$ & $(1)$                      & $(1\enspace4)(2\enspace3)$ & $(1\enspace3)(2\enspace4)$ \\
			$(1\enspace3)(2\enspace4)$ & $(1\enspace3)(2\enspace4)$ & $(1\enspace4)(2\enspace3)$ & $(1)$                      & $(1\enspace2)(3\enspace4)$ \\
			$(1\enspace4)(2\enspace3)$ & $(1\enspace4)(2\enspace3)$ & $(1\enspace3)(2\enspace4)$ & $(1\enspace2)(3\enspace4)$ & $(1)$                      \\
		\end{tabular} \\
	\bigskip

	{\noindent}Notice that each element of $\mathbf{V}$ is also an element of $S_4$. We proved in Exercise 1.3 that function composition, when applied to elements of $S_X$, is associative.
	We can then verify by inspection that:
	\begin{itemize}
		\item $\forall a,b\in\mathbf{V}$, $ab \in \mathbf{V}$
		\item $\forall a\in\mathbf{V}$, $ea=a=ae$ \quad[where $e$ is the element $(1)$]
		\item $\forall a\in\mathbf{V}$, $\exists b\in\mathbf{V}$ s.t. $ab=e=ba$ \quad[Specifically, $b=a$; \textsc{iow}, every element of $\mathbf{V}$ is its own inverse.]
	\end{itemize}
	Since set $\mathbf{V}$ with function composition as an operator is closed, associative, has an identity, and has inverses, it is a group.
	\end{proof}
\end{exercise}


\end{document}
