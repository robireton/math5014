\documentclass[12pt]{article}
\usepackage[top=1.0in, bottom=0.5in, left=0.75in, right=2.5in]{geometry}
\usepackage{fancyhdr, textpos}
\usepackage{amsmath, amssymb, amsthm}
\usepackage{ebgaramond, ebgaramond-maths}

\pagestyle{fancy}
\setlength\headheight{28pt}
\setlength\headwidth{7in}
\setlength{\TPHorizModule}{1pt}
\setlength{\TPVertModule}{1pt}
\lhead{\textsc{group theory\\spring 2018}}
\rhead{\textsc{homework bundle \#2\\page \thepage}}
\chead{\textsc{rob\\ireton}}

\newcommand{\zee}{\mathbb{Z}}

\newenvironment{exercise}[2]{\begin{textblock}{32}[1,0](0,#2)\noindent#1\end{textblock}}{\vspace{1in}}

\begin{document}
\begin{exercise}{1.16}{6.2}
	{\noindent}Let $\alpha\in S_n$ and $p\in \mathbb{N}$ be prime. Prove that if $\alpha^p=1$,
	then either $\alpha=1$, $\alpha$ is a $p$-cycle, or $\alpha$ is a product of disjoint $p$-cycles.
	\bigskip

\end{exercise}

\begin{exercise}{1.18}{4}
	{\noindent}Give an example of $\alpha, \beta, \gamma\in S_5$ such that:
	\begin{itemize}
		\item $\alpha$ commutes with $\beta$
		\item $\beta$ commutes with $\gamma$
		\item $\alpha$ and $\gamma$ do not commute.
	\end{itemize}
	\bigskip
	Based on the results of Exercise 1.8---that if $\alpha$ and $\beta$ are disjoint permutations,
	then they commute---choose $\alpha, \beta, \gamma\in S_5$ such that $\alpha, \beta$ disjoint,
	$\beta, \gamma$ disjoint, $\alpha, \gamma$ not disjoint. e.g.:
	$\alpha = (1\ 2)$; $\beta = (3\ 4)$; $\gamma = (1\ 5)$.

\end{exercise}

\begin{exercise}{1.21}{6.2}
  {\noindent}Show that $S_n$ has the same number of even permutations as odd permutations.
	[Hint: If $\tau$ is a transposition, consider the function $f : S_n\to S_n$ given by $f(\alpha)=\tau\alpha$.]
	\bigskip

\end{exercise}


\begin{exercise}{1.26}{3.5}
	{\noindent}Prove: a group $G$ in which $\forall x\in G$, $x^2=e$ must be abelian.
	\bigskip

\end{exercise}

\begin{exercise}{1.29}{6.2}
	{\noindent}Consider $\alpha: \zee_{11}\to \zee_{11}$ given by $\alpha(x)=4x^2-3x^7$.
	\begin{itemize}
		\item Show that $\alpha$ is a permutation of $\zee_{11}$ and write it as a product of disjoint cycles.
		\bigskip

		\item What is the parity of $\alpha$?
		\bigskip

		\item What is $\alpha^{-1}$?
		\bigskip

		$\alpha^{-1}=...$.
	\end{itemize}
\end{exercise}


\begin{exercise}{1.36}{6.2}
	{\noindent}Prove that the following four permutations form a group, denoted $\mathbf{V}$, called the (Klein) Four-group.
	\[(1);\; (1\ 2)(3\ 4);\; (1\ 3)(2\ 4);\; (1\ 4)(2\ 3).\]
	\bigskip

\end{exercise}


\begin{exercise}{1.39}{9}
	{\noindent}Let $f : X\to Y$ be a bijection between sets $X$ and $Y$. Show that the mapping given by \[\alpha \mapsto f\circ\alpha\circ f^{-1}\] is an isomorphism $S_x\to S_y$.
	\bigskip

\end{exercise}

\end{document}
