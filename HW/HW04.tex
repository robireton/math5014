\documentclass[12pt]{article}
\usepackage[top=1.0in, bottom=0.75in, left=0.75in, right=2.5in]{geometry}
\usepackage{fancyhdr, textpos, faktor}
\usepackage{amsmath, amssymb, amsthm}
\usepackage{ebgaramond-maths, plex-mono}

\pagestyle{fancy}
\fancyhf{}
\setlength\headheight{28pt}
\setlength\headwidth{7in}
\setlength{\TPHorizModule}{1pt}
\setlength{\TPVertModule}{1pt}
\lhead{\textsc{group theory\\spring 2018}}
\rhead{\textsc{homework bundle \#4\\page \thepage}}
\chead{\textsc{rob\\ireton}}

\newcommand{\zee}{\mathbb{Z}}

\newenvironment{exercise}[2]{\begin{textblock}{32}[1,0](0,#2)\noindent#1\end{textblock}}{\vspace{1in}}
\newtheorem*{lemma}{Lemma}

\begin{document}

% \begin{exercise}{2.23}{6}
% 	{\noindent}If $\gcd(r,s)=1$, then $s^{\varphi(r)}\equiv 1 \mod r$.
% 	[\textit{Hint}. The order of the group of units $U(\zee_n)$ is $\varphi(n)$.]
% 	\bigskip
%
% 	$r$ and $s$ are relatively prime.\\
% 	$\varphi(r)$ is the number of positive integers that are relatively prime to $r$.\\
% 	Need to show that $r$ divides $s^{\varphi(r)}-1$.\\
% 	\dots
% \end{exercise}

% \begin{exercise}{2.26}{6}
% 	{\noindent}Let $\{S_i;\; i\in I\}$ be a family of subgroups of a group $G$, let $\{S_it_i;\; i\in I\}$ be a family of right cosets, and let $D=\bigcap_{i\in I}S_i$.  Prove that either $\bigcap_{i\in I}S_it_i=\emptyset$ or $\bigcap_{i\in I}S_it_i=Dg$ for some $g\in G$.
% 	\bigskip
%
% \end{exercise}

\begin{exercise}{2.30}{6}
	{\noindent}If $S\le G$ and $[G:S]=2$, then $S\vartriangleleft G$.
	\bigskip

	\begin{proof}
		There are two right cosets of $S$ in $G$: $\{Sg_1, Sg_2\}$, with $g_1, g_2$ representatives of their respective right cosets.
		We can take $g_1$ to be $e$. $Se=eS$, so those left and right cosets are the same. $Sg_2=G-Se=G-eS=g_2S$, so the other left and right cosets are the same as well.
		Since the left cosets of $S$ are the same as the right cosets of $S$, $S\vartriangleleft G$.
	\end{proof}


\end{exercise}

\begin{exercise}{2.36}{6}
	{\noindent}Prove that $A_n\vartriangleleft S_n$ for every $n\ge 2$.
	\bigskip

	\begin{proof}
		$A_n$, the set of all even permutations in $S_n$ is a group with order $n!/2$. (ex 2.1)\\
		Any particular $S_n$ is finite, so $|S_n|/|A_n|=[S_n:A_n]=2$ (Lagrange).\\
		$[S_n:A_n]=2$, so $A_n\vartriangleleft S_n$ by exercise 2.30 above.
	\end{proof}
\end{exercise}

% \newpage

\begin{exercise}{2.40}{6}
	{\noindent}Let $H\vartriangleleft G$, let $\nu:G\to \faktor{G}{H}$ be the natural map, and let $X\subset G$ be a subset such that $\nu(X)$ generates $\faktor{G}{H}$. Prove that $G=\langle H\cup X\rangle$.
	\bigskip

	{\noindent}$\forall h\in H, \forall g\in G, ghg^{-1}\in H$.\\
	$\nu: g\mapsto Hg$.\\
	$\faktor{G}{H}=\langle\nu(X)\rangle=\langle\{Hx; x\in X\}\rangle$\\
	WTS: every element of $G$ can be generated by and element of $H$ or by an element of $X$.
	\dots

	\begin{proof}
		$H\vartriangleleft G$, so the cosets of $H$ partition $G$. Those cosets are generated by $\nu(X)$, so every element of $G$ is in a coset of $H$, which is also a subgroup
	\end{proof}
\end{exercise}

% \begin{exercise}{2.41}{6}
% 	{\noindent}Let $G$ be a finite group of odd order, and let $x$ be the product of all the elements of $G$ (in some order). Prove that $x\in G'$.
% 	\bigskip
%
% 	\begin{lemma}
% 		Let $x\in G$, $x\neq e$. Suppose that $x=x^{-1}$. Then $\{e, x\}\leq G$. But $|\{e, x\}|=2$ and $2\nmid |G|$. (Recall that $|G|$ is odd.)
% 		Since Lagrange's Theorem requires that the order of a subgroup divide the order of the group, we have a contradiction, so if $|G|$ odd, $x\in G$, $x\neq e$, then $x\neq x^{-1}$.
% 	\end{lemma}
%
% 	{\noindent}cosets of $G'$ ($\faktor{G}{G'}$)\\
% 	WTS: $G'x=G'$\\
% 	Theorem 2.23\\
%
% \end{exercise}

\end{document}
