\documentclass[12pt]{article}
\usepackage[top=1.0in, bottom=0.75in, left=0.75in, right=2.5in]{geometry}
\usepackage{fancyhdr, textpos, faktor, enumerate}
\usepackage{amsmath, amssymb, amsthm}
\usepackage{ebgaramond-maths, plex-mono}

\pagestyle{fancy}
\fancyhf{}
\setlength\headheight{28pt}
\setlength\headwidth{7in}
\setlength{\TPHorizModule}{1pt}
\setlength{\TPVertModule}{1pt}
\lhead{\textsc{group theory\\spring 2018}}
\rhead{\textsc{homework bundle \#5\\page \thepage}}
\chead{\textsc{rob\\ireton}}

\newenvironment{exercise}[2]{\begin{textblock}{32}[1,0](0,#2)\noindent#1\end{textblock}}{\vspace{1in}}

\begin{document}

\begin{exercise}{1.27}{6}
  \begin{enumerate}[(i.)]
  	\item Let $G$ be a finite abelian group containing no elements $a\ne e$ with $a^2=e$. Evaluate $a_1*a_2*\dotsb*a_n$ (a list of all the elements of $G$ with no repetitions).
    \item Prove Wilson's Theorem: If $p$ is prime, then $(p-1)!\equiv -1\mod p$.
  \end{enumerate}
\end{exercise}

\begin{exercise}{1.46}{6}
  {\noindent}Let $\boldsymbol{T}$ denote the circle group: the multiplicative group of all complex numbers of absolute value 1. For a fixed real number $y$, show that $f_y:\mathbb{R}\to \boldsymbol{T}$, given by $f_y(x)=e^{iyx}$, is a group homomorphism.

  \begin{proof}
    Let $a,b\in\mathbb{R}$. Then $f_y(a+b)=e^{iy(a+b)}=e^{iya}e^{iyb}=f_y(a)f_y(b)$.\\
    Therefore, $f_y$ is a group homomorphism.
  \end{proof}
\end{exercise}

\begin{exercise}{2.20}{6}
	{\noindent}Prove: If $G=\langle a \rangle$ is a cyclic group of order $n$, then $a^k$ is also a generator of $G$ $\iff$ $gcd(n,k)=1$.
\end{exercise}

\begin{exercise}{2.46}{6}
	{\noindent}Let $N\vartriangleleft G$ and let $f: G\to H$  be a homomorphism whose kernel contains $N$. Show that $f$ induces a homomorphism $f_*:\faktor GN\to H$ by $f_*(Na)=f(a)$.
\end{exercise}

\begin{exercise}{2.49}{6}
	{\noindent}Let $H,K,L$ be subgroups of $G$ with $H\le L$. Prove $HK\cap L=H(K\cap L)$. (We do not assume that either $HK$ or $H(K\cap L)$ is a subgroup.)
\end{exercise}

\begin{exercise}{2.50}{6}
	{\noindent}Let $f: G\to G^*$ be a homomorphism and let $S^*\le G^*$. Prove that $f^{-1}(S^*)=\{ x\in G;\; f(x)\in S^*\}$ is a subgroup of $G$ containing $ker\ f$. [NOTE: In this context $f^{-1}$ is used to denote ``preimage'', it is not assumed that $f$ has an inverse.]
\end{exercise}

\begin{exercise}{2.56}{6}
	{\noindent}A subgroup $H\le G$ is a maximal normal subgroup of $G$ if there is no normal subgroup of $G$ with $H<N<G$. Prove that $H$ is a maximal normal subgroup of $G$ $\iff$ $\faktor GH $ has no normal subgroups (other than itself and $\{e\}$).
\end{exercise}

\begin{exercise}{2.58}{6}
	{\noindent}Let $M$ be a maximal subgroup of $G$. Prove that of $M\vartriangleleft G$, then $[G:M]$ is finite and equal to a prime.
\end{exercise}

\begin{exercise}{2.69}{6}
	{\noindent}Let $G$ be a group with normal subgroups $H$ and $K$. Prove that: $HK=G$ and $H\cap  K=\{e\}$ $\iff$ each $a\in G$ has a unique expression of the form $a=hk$ where $h\in H$ and $k\in K$.
\end{exercise}

\begin{exercise}{2.72}{6}
	{\noindent}Let $G$ be a group having a simple subgroup $H$ with index 2. Prove that either:
  \begin{enumerate}[(a.)]
    \item $H$ is the unique proper normal subgroup of $G$, OR
	  \item $\exists K\le G$ with $|K|=2$ such that $G=H\times K$.
  \end{enumerate}
\end{exercise}

\begin{exercise}{2.74}{6}
	{\noindent}Prove: The operation of direct product is commutative and associative in the following sense: for groups $H$, $K$, and $L$, $H\times K\cong K\times H$ and $H\times(K\times L)=(H\times K)\times L$.

  \begin{proof}
    Let $\varphi:H\times K\to K\times H$ by $\varphi((h,k)) = (k,h)$; $h,h'\in H$; $k,k'\in K$.
    $\varphi((h,k)(h',k')) = \varphi((hh',kk')) = (kk',hh') = (k,h)(k',h') = \varphi((h,k))\varphi((h',k'))$. Therefore, $\varphi$ is a homomorphism.
    Next, suppose that $\varphi((h,k))=\varphi((h',k'))$. Then $(k,h)=(k',h')$, $k=k'$ and $h=h'$, so $(h,k)=(h',k')$, meaning that $\varphi$ is injective.
    Now let $(k,h)$ be an element of $K\times H$. Then there exists an element: $(h,k)$, which is in $H\times K$ and $\varphi((h,k))=(k,h)$. So $\varphi$ is surjective, too.
    Since there is a bijective homomorphism between $H\times K$ and $K\times H$, we can say that $H\times K\cong K\times H$.\\
    \\
    \textit{The book asks for a proof that $H\times(K\times L)\cong(H\times K)\times L$, that is, isomorphic instead of equal, as is specified here. I'm going to discuss `equal' below, but let us know if you meant `isomorphic'.}\\
    \\
    Arguably, the most natural way to interpret both $H\times(K\times L)$ and $(H\times K)\times L$ is as specifying an
    ordered triplet with elements of the form $(h,k,l)$ where $h\in H$, $k\in K$, and $l\in L$. This is a group for reasons
    similar to why an external direct product is a group. The operation is $(h,k,l)(h',k',l')=(hh',kk',ll')$, the identity is
    $(1,1,1)$, and the inverse of $(h,k,l)$ is $(h^{-1},k^{-1},l^{-1})$. $H\times(K\times L)$ and $(H\times K)\times L$ specify
    the same group of ordered triplets, so they are equal.
  \end{proof}
\end{exercise}

\end{document}
