\documentclass[12pt]{article}
\usepackage[top=1.0in, bottom=0.75in, left=0.75in, right=2.5in]{geometry}
\usepackage{fancyhdr, textpos, faktor, enumerate}
\usepackage{amsmath, amssymb, amsthm}
\usepackage{ebgaramond-maths, plex-mono}

\pagestyle{fancy}
\fancyhf{}
\setlength\headheight{28pt}
\setlength\headwidth{7in}
\setlength{\TPHorizModule}{1pt}
\setlength{\TPVertModule}{1pt}
\lhead{\textsc{group theory\\spring 2018}}
\rhead{\textsc{homework bundle \#6\\page \thepage}}
\chead{\textsc{rob\\ireton}}

\newenvironment{exercise}[2]{\begin{textblock}{32}[1,0](0,#2)\noindent#1\end{textblock}}{\vspace{1in}}

\begin{document}
  \begin{exercise}{3.1}{6}
    {\noindent}A group $G$ is centerless if $Z(G)=\{e\}$. Prove that $S_n$ is centerless if $n\ge 3$.

    \begin{proof}
      Suppose that $\exists\alpha\in Z(S_n), n\geq 3$, such that $\alpha\neq e$.
      Then $\exists x,y\in\{1,\ldots,n\}, x\neq y$, such that $\alpha(x)=y$.
      Since $n\geq 3$, there must also be a $z$ in $\{1,\ldots,n\}$ that is distinct from both $x$ and $y$.\\
      Consider $\beta = (x\enspace z) \in S_n$.
      $\beta\alpha(x) = \beta(y) = y$.
      $\alpha\beta(x) = \alpha(z)$.
      Since $\alpha$ is a permutation, we know that $\alpha(z)\neq\alpha(x)$, so $\alpha(z)\neq y$.
      However, $\beta\alpha(a)\neq\alpha\beta(a)$ contradicts $\alpha\in Z(S_n)$.\\
      Therefore, $S_n$ must be centerless for $n\geq 3$.
    \end{proof}
  \end{exercise}

  \begin{exercise}{3.3}{6}
    {\noindent}Prove that if $G$ is not abelian, then $\faktor{G}{Z(G)}$ is not cyclic.

    \begin{proof}
      Group $G$ is not abelian, so $\exists\alpha,\beta\in G$ such that $\alpha\beta\neq\beta\alpha$.
      Observe that $\alpha,\beta\not\in Z(G)$, so the conjugacy class of $\alpha$ contains more than just $\alpha$; likewise for $\beta$.\\\
      \dots
      \\
      Show that $\faktor{G}{Z(G)}$ must be generated by more than one element.
    \end{proof}
  \end{exercise}

  % \begin{exercise}{3.5}{6}
  %   {\noindent}Prove that $Z(G_1\times\dotsb\times G_n)=Z(G_1)\times\dotsb\times Z(G_n)$.
  % \end{exercise}
  %
  % \begin{exercise}{3.6\,i}{6}
  %   {\noindent}Prove that $\forall a,x\in G$, $C_G(axa^{-1})=aC_G(x)a^{-1}$.
  % \end{exercise}
  %
  % \begin{exercise}{3.9\,i}{6}
  %   {\noindent}Prove that $N_G(aHa^{-1})=aN_G(H)a^{-1}$.
  % \end{exercise}
  %
  % \begin{exercise}{3.9\,ii}{6}
  %   {\noindent}If $H\le K\le G$< prove that $N_K(H)=N_G(H)\cap K$.
  % \end{exercise}
  %
  % \newpage
  %
  % \begin{exercise}{3.14\,i}{6}
  %   {\noindent}Prove that if the conjugacy class of $x\in G$ is $\{a_1,\dotsc,a_k\}$, then the conjugacy class of $x^{-1}$ is $\{a_1^{-1},a_2^{-1},\dotsc, a_k^{-1}\}$.
  % \end{exercise}
  %
  % \begin{exercise}{3.16}{6}
  %   {\noindent}Show that the number of conjugacy classes in $S_n$ is the number of partitions of $n$.
  % \end{exercise}
  %
  % \begin{exercise}{3.20}{6}
  %   {\noindent}Show that $A_5$, a group of order 60, has no subgroup of order 30.
  %
  %   \begin{proof}
  %     $A_5$ is simple by theorem 3.8.
  %   \end{proof}
  % \end{exercise}
  %
  % \begin{exercise}{3.26\,i}{6}
  %   {\noindent}If $\rho$ is the representation of a group $G$ on the cosets of a subgroup $H$, then $\ker \rho=\displaystyle\bigcap_{x\in G}xHx^{-1}$.
  % \end{exercise}
  %
  % \begin{exercise}{3.26\,ii}{6}
  %   {\noindent}If $\psi$ is the representation of a group $G$ on the conjugates of a subgroup  $H$, then $\ker \psi=\displaystyle\bigcap_{x\in G}xN_G(H)x^{-1}$.
  % \end{exercise}
  %
  % \begin{exercise}{3.27}{6}
  %   {\noindent}The right regular representation of a group $G$ is the function $R:\, G\to S_G$ defined by $a\mapsto R_a$, where $R_a(x)=xa^{-1}$. Show that $R$ is an injective homomorphism.
  %   [\textit{Note: in class we discussed the LEFT regular representation, this problem is about the RIGHT representation.}]
  % \end{exercise}
\end{document}
