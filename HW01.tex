\documentclass[12pt]{article}
\usepackage[top=1.0in, bottom=0.5in, left=0.75in, right=2.5in]{geometry}
\usepackage{fancyhdr, fourier-orns, tabularx, lipsum}
\usepackage{amsmath, amssymb, amsthm}

\usepackage{ebgaramond, ebgaramond-maths}


%\usepackage{graphicx,tikz}
%\usepackage{enumerate}
\pagestyle{fancy}
\setlength\headheight{38pt}
\setlength\headwidth{7in}
\lhead{\textsc{group theory\\spring 2018}}
\rhead{\textsc{homework bundle \#1\\page \thepage}}
\chead{\textsc{rob\\ireton}}
\renewcommand\headrule{\hrulefill{\enskip\leafright\decosix\leafleft\enskip}\hrulefill{\enskip\leafright\decosix\leafleft\enskip}\hrulefill}

\usepackage{titlesec}
\newlength\titleindent
\setlength\titleindent{1in}
\titleformat{\section}
  {\normalfont\Large\bfseries}{\llap{\parbox{\titleindent}{\thesection\hfill}}}{0em}{}


\newcommand{\zee}{\mathbb{Z}}

\newenvironment{exercises}
{\noindent\begin{tabularx}{\textwidth}{rX}}
{\end{tabularx}}

\newenvironment{exercise}[1]
{\Large#1 &}
{\vspace{1in}\quad \\}

\begin{document}

% \noindent
% \begin{tabularx}{\textwidth}{rX}
%   \Large{1.0} & \lipsum[1]\vspace{1in}\quad \\
%   \Large{1.1} & \lipsum[2] \\
%   \Large{1.23i} & \lipsum[3] \\
% \end{tabularx}

% \begin{exercise}{1.3}
% \begin{itemize}
% 	\item Let $X$ be a nonempty set. Prove that $\forall\, \alpha, \beta, \gamma\in \, S_X$, $\alpha(\beta\gamma)=(\alpha\beta)\gamma$.
% 	\bigskip
%
% 	put your proof here
%
% 	\vspace{1in}
%
% 	\item Let $X,Y,Z,W$ be sets and suppose $f:X\to Y,\; g:Y\to Z,\; h:Z\to W$ are functions. Prove that $h(gf)=(hg)f$.
%
% 		\bigskip
%
% 	put your proof here
%
%
% \end{itemize}
% \end{exercise}
%

% \begin{exercises}
% 	\begin{exercise}{1.6}
% 		Prove the cancellation law for permutations: If $\alpha, \beta, \gamma$ are permutations and either $\alpha\beta=\alpha\gamma$ or $\beta\alpha=\gamma\alpha$, then $\beta=\gamma$.
%
% 			\bigskip
%
% 			put your proof here
%
% 	\end{exercise}
% \end{exercises}



\section*{1.8} Prove that if $\alpha$ and $\beta$ are disjoint permutations, then $\alpha\beta=\beta\alpha$, that is, $\alpha$ and $\beta$ commute.

\bigskip

	put your proof here

	\vspace{1in}

\section*{1.10} Let $\alpha, \beta\in S_n$. Define $\alpha^0=1$, $\alpha^1=\alpha$, and for $k\ge 2$ $\alpha^k$ is the composition of $\alpha$ with itself $k$ times.

\begin{itemize}
	\item Prove that if $\alpha$ and $\beta$ are disjoint, then $\forall k\ge0$, $(\alpha\beta)^k=\alpha^k\beta^k$

	\bigskip

	put your proof here

	\vspace{1in}

\item If $\alpha$ and $\beta$ are not disjoint, is the result still true? Justify your answer.

	\bigskip

	put your proof here

	\vspace{1in}

\end{itemize}

\section*{1.16} Let $\alpha\in S_n$ and $p\in \mathbb{N}$ be prime. Prove that if $\alpha^p=1$, then either $\alpha=1$, $\alpha$ is a $p$-cycle, or $\alpha$ is a product of disjoint $p$-cycles.

	\bigskip

	put your proof here

	\vspace{1in}

\section*{1.18} Give an example of $\alpha. \beta, \gamma\in S_5$ such that:
\begin{itemize}
	\item $\alpha$ commutes with $\beta$
	\item $\beta$ commutes with $\alpha$
	\item $\alpha$ and $\gamma$ do not commute.
\end{itemize}

\bigskip

	put your answer here

	\vspace{1in}

\section*{1.21} Show that $S_n$ has the same number of even permutations as odd permutations.

[Hint: If $\tau$ is a transposition, consider the function $f : S_n\to S_n$ given by $f(\alpha)=\tau\alpha$.]
\end{document}
