\documentclass[12pt]{article}
\usepackage[top=1.0in, bottom=0.5in, left=0.75in, right=2.5in]{geometry}
\usepackage{fancyhdr, textpos}
\usepackage{amsmath, amssymb, amsthm}
\usepackage{ebgaramond, ebgaramond-maths}

\pagestyle{fancy}
\setlength\headheight{28pt}
\setlength\headwidth{7in}
\setlength{\TPHorizModule}{1pt}
\setlength{\TPVertModule}{1pt}
\lhead{\textsc{group theory\\spring 2018}}
\rhead{\textsc{homework bundle \#1\\page \thepage}}
\chead{\textsc{rob\\ireton}}

\newcommand{\zee}{\mathbb{Z}}

\newenvironment{exercise}[2]{\begin{textblock}{32}[1,0](0,#2)\noindent#1\end{textblock}}{\vspace{1in}}

\begin{document}

\begin{exercise}{1.3}{9.5}
	{\noindent}Let $X$ be a nonempty set. Prove that $\forall\, \alpha, \beta, \gamma\in \, S_X$, $\alpha(\beta\gamma)=(\alpha\beta)\gamma$.
	\bigskip

	\begin{proof}
    Let $x\in X$. Let $\alpha,\beta,\gamma\in S_X$.\\
		\\
		$(\alpha(\beta\gamma))(x) = (\alpha(\beta\gamma(x))) = \alpha(\beta(\gamma(x))) = \alpha\beta(\gamma(x)) = ((\alpha\beta)\gamma)(x)$.\\
		\\
		Since the values of the functions $\alpha(\beta\gamma)$ and $(\alpha\beta)\gamma$ are the same for all elements in X, the functions are equal.
  \end{proof}
	\vspace{1in}

	{\noindent}Let $X,Y,Z,W$ be sets and suppose $f:X\to Y,\; g:Y\to Z,\; h:Z\to W$ are functions. Prove that $h(gf)=(hg)f$.
	\bigskip

	\begin{proof}
		Let $x\in X$.\\
		\\
		$(h(gf))(x) = h(gf(x)) = h(g(f(x))) = hg(f(x)) = ((hg)f)(x)$.\\
		\\
		Since the values of the functions $h(gf)$ and $(hg)f$ are the same for all elements in X, the functions are equal.
	\end{proof}
\end{exercise}


\begin{exercise}{1.6}{3.5}
	{\noindent}Prove the cancellation law for permutations: If $\alpha, \beta, \gamma$ are permutations and either $\alpha\beta=\alpha\gamma$ or $\beta\alpha=\gamma\alpha$, then $\beta=\gamma$.
	\bigskip

	\begin{proof}
		Observe that $\alpha$ is a bijective function, so $\alpha^{-1}$ exists and $\alpha\alpha^{-1} = 1 = \alpha^{-1}\alpha$.
		\\
		\\
		If $\alpha\beta=\alpha\gamma$, then $\alpha^{-1}\alpha\beta=\alpha^{-1}\alpha\gamma$ and $1\beta=1\gamma$. $1$ is the identity function, so $\beta=\gamma$.
		\\
		\\
		If $\beta\alpha=\gamma\alpha$, then $\beta\alpha\alpha^{-1}=\gamma\alpha\alpha^{-1}$ and $\beta 1=\gamma 1$. Again, $\beta=\gamma$.
	\end{proof}
\end{exercise}

\newpage

\begin{exercise}{1.8}{3.5}
	{\noindent}Prove that if $\alpha$ and $\beta$ are disjoint permutations, then $\alpha\beta=\beta\alpha$, that is, $\alpha$ and $\beta$ commute.
	\bigskip

	\begin{proof}[Lemma]
		Let $X$ be a set and $f:X\to X$ bijective. Suppose to the contrary that $f$ moves some $x\in X$, but fixes $f(x)$. Then we have $f(x) = f(f(x))$ because $f$ moves $x$ but fixes $f(x)$; but we also have $x\neq f(x)$ because $f$ moves $x$. This conflicts with $f$ being injective.
		So, if $f$ moves $x$, then $f$ also moves $f(x)$.
	\end{proof}
	\bigskip

	\begin{proof}
		If $\alpha$ and $\beta$ are disjoint permutations, then they are both bijective functions from a set $X$ to itself. Since they are disjoint, for each $x\in X$, if $\alpha$ moves $x$, then $\beta$ fixes it, and if $\beta$ moves $x$, then $\alpha$ fixes it.\\
		\\
		In the case where $\alpha$ moves $x$, $\beta$ fixes $x$ so $\beta(x) = x$ and $\alpha(\beta(x)) = \alpha(x)$.\\
		Since $\alpha$ moves $x$, it also moves $\alpha(x)$ (see lemma), so $\beta$ fixes $\alpha(x)$ so $\beta(\alpha(x)) = \alpha(x)$.\\
		\\
		By a similar argument, when $\alpha$ fixes $x$, $\alpha(\beta(x)) = \beta(x) = \beta(\alpha(x))$.\\
		\\
		Therefore, when $\alpha$ and $\beta$ are disjoint permutations, they commute.
	\end{proof}
\end{exercise}

\newpage

\begin{exercise}{1.10}{6.2}
  {\noindent}Let $\alpha, \beta\in S_n$. Define $\alpha^0=1$, $\alpha^1=\alpha$, and for $k\ge 2$ $\alpha^k$ is the composition of $\alpha$ with itself $k$ times.
  \begin{itemize}
    \item Prove that if $\alpha$ and $\beta$ are disjoint, then $\forall k\ge0$, $(\alpha\beta)^k=\alpha^k\beta^k$
    \bigskip

  	\begin{proof}
			From the definition, when $k=0$, we have $(\alpha\beta)^0 = 1 = 1\circ 1 = \alpha^0\beta^0$.
			Likewise, when $k=1$, $(\alpha\beta)^1 = \alpha\beta = \alpha^1\beta^1$.\\
			\\
			Now, suppose that $(\alpha\beta)^n = \alpha^n\beta^n$.
			Then, $(\alpha\beta)(\alpha\beta)^n = (\alpha\beta)^{n+1} = (\alpha\beta)(\alpha^n\beta^n)$.
			$\alpha$ and $\beta$ are disjoint, so we know from exercise 1.8 that they commute, allowing us to write
			$(\alpha\beta)(\alpha^n\beta^n)=\alpha\beta\alpha^n\beta^n=\alpha\alpha^n\beta\beta^n=\alpha^{n+1}\beta^{n+1}$.
			This gives $(\alpha\beta)^{n+1} = \alpha^{n+1}\beta^{n+1}$.\\
			\\
			We know that $(\alpha\beta)^k=\alpha^k\beta^k$ holds for $k=0$ and $k=1$, so, by induction, it must hold for every $k>1$ as well.
  	\end{proof}
  	\vspace{1in}

    \item If $\alpha$ and $\beta$ are not disjoint, is the result still true? Justify your answer.
	  \bigskip

		The result is \textit{not} still true.\\
		\\
		Take $\alpha\in S_3 = (1\ 2\ 3)$ and $\beta\in S_3 = (1\ 3)$, which are not disjoint. $\alpha\beta = (2\ 3)$ and $(\alpha\beta)^2 = 1$. $\alpha^2 = (1\ 3\ 2)$. $\beta^2 = 1$.
		$(\alpha\beta)^2 \neq \alpha^2\beta^2$, so this is a countexample.
  \end{itemize}
\end{exercise}

\end{document}
